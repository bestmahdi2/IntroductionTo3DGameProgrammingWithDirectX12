\title{
    \center \Huge
    معرفی \\[25pt]
}

{ \large

{
    Direct3D 12
    یک کتابخانه رندر برای نوشتن برنامه های گرافیکی سه بعدی با کارایی بالا با استفاده از سخت افزار گرافیکی مدرن بر روی پلتفرم های مختلف ویندوز 10
    (Windows Desktop - Mobile - Xbox One)
    است.
    Direct3D
    یک کتابخانه سطح پایین است به این معنا که رابط برنامه نویسی کاربردی آن (API) سخت افزار گرافیکی زیرینی را که کنترل می کند ، مدل سازی می کند.
مصرف کننده اصلی Direct3D صنعت بازی است که در آن موتورهای رندر سطح بالاتر بر روی Direct3D ساخته می شوند.
با این حال، صنایع دیگر به گرافیک سه بعدی تعاملی با کارایی بالا نیز نیاز دارند، مانند تجسم پزشکی و علمی و بررسی های معماری.
علاوه بر این، با مجهز شدن هر رایانه شخصی جدید به یک کارت گرافیک مدرن، برنامه های غیر سه بعدی شروع به استفاده از GPU (واحد پردازش گرافیکی) برای تخلیه کار به کارت گرافیک برای محاسبات فشرده می کنند.
این به عنوان محاسبات GPU با هدف عمومی شناخته می شود و Direct3D API سایه زن محاسباتی را برای نوشتن برنامه های GPU با هدف عمومی ارائه می دهد.
اگرچه Direct3D 12 معمولاً از C++ native برنامه‌ریزی می‌شود، تیم
SharpDX (http://sharpdx.org/)
    در حال کار بر روی بسته های .NET هستند تا بتوانید از برنامه‌های مدیریت شده به این API گرافیکی سه بعدی قدرتمند دسترسی داشته باشید.
}

{این کتاب مقدمه ای بر برنامه نویسی گرافیک های کامپیوتری تعاملی با تاکید بر توسعه بازی با استفاده از Direct3D 12 ارائه می دهد. اصول برنامه نویسی Direct3D و shader را آموزش می دهد و پس از آن خواننده آماده می شود تا تکنیک های پیشرفته تری را یاد بگیرد. کتاب به سه بخش اصلی تقسیم شده است. بخش اول ابزارهای ریاضی را توضیح می دهد که در سراسر این کتاب استفاده خواهد شد. بخش دوم نحوه پیاده سازی وظایف اساسی در Direct3D مانند مقداردهی اولیه را نشان می دهد. تعریف هندسه سه بعدی؛ راه اندازی دوربین ها؛ ایجاد رئوس، پیکسل، هندسه، و محاسبه سایه زن. نورپردازی؛ بافت سازی؛ مخلوط کردن؛ شابلون سازی; و تسلیت. بخش سوم عمدتاً در مورد استفاده از Direct3D برای پیاده سازی انواع تکنیک های جالب و جلوه های ویژه است، مانند کار با مش شخصیت های متحرک، برداشتن، نگاشت محیط، نقشه برداری معمولی، سایه های بلادرنگ، و انسداد محیط.}

{برای مبتدیان، این کتاب بهتر است جلو به عقب بخواند. فصل ها به گونه ای سازماندهی شده اند که با هر فصل، مشکل به تدریج افزایش می یابد. به این ترتیب، هیچ جهش ناگهانی در پیچیدگی وجود ندارد که خواننده را گم کند. به طور کلی، برای یک فصل خاص، از تکنیک ها و مفاهیمی که قبلاً توسعه داده شده است استفاده خواهیم کرد. بنابراین، مهم است که قبل از ادامه، بر مطالب یک فصل تسلط داشته باشید. خوانندگان با تجربه می توانند فصل های مورد علاقه خود را انتخاب کنند. در نهایت، ممکن است از خود بپرسید که پس از خواندن این کتاب چه نوع بازی هایی را می توانید توسعه دهید. پاسخ این سوال را بهتر است با مرور این کتاب و مشاهده انواع برنامه های توسعه یافته به دست آورید. از این رو باید بتوانید انواع بازی هایی را که می توان بر اساس تکنیک های آموزش داده شده در این کتاب و برخی از نبوغ خود توسعه داد، تجسم کنید.}
}
%--------------------------------------%
\title{
    \LARGE
    مخاطب مورد نظر
}
\\ \rule{\textwidth}{0.5pt}
{این کتاب با در نظر گرفتن سه مخاطب زیر طراحی شده است:}
\begin{enumerate}
    \item {برنامه نویسان سطح متوسط C++ که می خواهند مقدمه ای بر برنامه نویسی سه بعدی با استفاده از آخرین نسخه}
    \item {برنامه نویسان سه بعدی با یک API غیر از DirectX (به عنوان مثال، OpenGL) تجربه کرده اند و می خواهند یک مقدمه برای Direct3D 12 داشته باشند.}
    \item {برنامه نویسان با تجربه Direct3D که مایل به یادگیری آخرین نسخه Direct3D هستند}
\end{enumerate}

\title{
    \LARGE
    پیش نیازها
}
{لازم به ذکر است که این مقدمه ای بر Direct3D 12، برنامه نویسی سایه زن و برنامه نویسی بازی های سه بعدی است. این مقدمه ای برای برنامه نویسی کامپیوتر عمومی نیست. خواننده باید شرایط زیر را رعایت کند:}

\begin{enumerate}
    \item {ریاضیات دبیرستان: مثلاً جبر، مثلثات و توابع (ریاضی).}
    \item {صلاحیت با ویژوال استودیو: باید بداند که چگونه پروژه ها را ایجاد کند، فایل ها را اضافه کند و کتابخانه های خارجی را برای پیوند مشخص کند.}
    \item {C++ متوسط و مهارت های ساختار داده: به عنوان مثال، با اشاره گرها، آرایه ها، بارگذاری بیش از حد اپراتورها، لیست های پیوندی، وراثت و چندشکلی راحت است.}
    \item {آشنایی با برنامه نویسی ویندوز با Win32 API مفید است، اما لازم نیست. ما یک پرایمر Win32 را در ضمیمه A ارائه می دهیم.}
\end{enumerate}

\title{
    \LARGE
    ابزارها و سخت افزارهای توسعه مورد نیاز
}
{برای برنامه نویسی برنامه های Direct3D 12 به موارد زیر نیاز است:}
\begin{enumerate}
    \item {Windows 10.}
    \item {ویژوال استودیو 2015 یا جدیدتر.}
    \item {کارت گرافیکی که از Direct3D 12 پشتیبانی می کند. دموهای این کتاب روی Geforce GTX 760 تست شده اند.}
\end{enumerate}

\newpage
%--------------------------------------%
