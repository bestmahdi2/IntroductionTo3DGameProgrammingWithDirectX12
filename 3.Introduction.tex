\title{
    \center \Huge
    معرفی \\[25pt]
}

{
        {  \large
    \lr{Direct3D 12} یک کتابخانه رندر برای نوشتن برنامه های گرافیکی سه بعدی با کارایی بالا با استفاده از سخت افزار گرافیکی مدرن بر روی پلتفرم های مختلف ویندوز 10 \lr{(Windows Desktop - Mobile - Xbox One)} است.
    \lr{Direct3D} یک کتابخانه سطح پایین است به این معنا که رابط برنامه نویسی کاربردی آن (API) سخت افزار گرافیکی زیرینی را که کنترل می کند ، مدل سازی می کند.
    مصرف کننده اصلی \lr{Direct3D} صنعت بازی است که در آن موتورهای رندر سطح بالاتر بر روی \lr{Direct3D} ساخته می شوند.
    با این حال، صنایع دیگر به گرافیک سه بعدی تعاملی با کارایی بالا نیز نیاز دارند، مانند تجسم پزشکی و علمی و بررسی های معماری.
    علاوه بر این، با مجهز شدن هر رایانه شخصی جدید به یک کارت گرافیک مدرن، برنامه های غیر سه بعدی شروع به استفاده از GPU (واحد پردازش گرافیکی) برای تخلیه کار به کارت گرافیک برای محاسبات فشرده می کنند.
    این کار به عنوان \lr{\textit{general purpose GPU computing}} شناخته می شود و \lr{Direct3D API} ، shader محاسباتی را برای نوشتن برنامه های GPU با هدف عمومی ارائه می دهد.
    اگرچه \lr{Direct3D 12} معمولاً با \lr{native C++}  برنامه نویسی می‌شود، تیم  \lr{\href{http://sharpdx.org/}{SharpDX}} در حال کار بر روی بسته های .NET هستند تا بتوانید از برنامه‌های مدیریت شده به این API گرافیکی سه بعدی قدرتمند دسترسی داشته باشید.
    }

        {  \large
    این کتاب مقدمه ای بر برنامه نویسی گرافیک کامپیوتری تعاملی با تاکید بر توسعه بازی با استفاده از \lr{Direct3D 12} ارائه می دهد. اصول برنامه نویسی \lr{Direct3D} و \lr{shader} را آموزش می دهد و پس از آن خواننده آماده می شود تا تکنیک های پیشرفته تری را یاد بگیرد.
    کتاب به سه بخش اصلی تقسیم شده است. بخش اول ابزار های ریاضی را توضیح می دهد که در سراسر این کتاب استفاده خواهد شد.
    بخش دوم نحوه پیاده سازی وظایف اساسی در \lr{Direct3D} مانند مقداردهی اولیه، تعریف هندسه سه بعدی؛ راه اندازی دوربین ها؛ ایجاد رئوس، پیکسل، هندسه، و محاسبه سایه زن ، نورپردازی؛ بافت سازی؛ مخلوط کردن؛ شابلون سازی; و تسلیت را نشان می دهد.
    بخش سوم عمدتاً در مورد استفاده از \lr{Direct3D} برای پیاده سازی انواع تکنیک های جالب و جلوه های ویژه است، مانند کار با مش شخصیت های متحرک، برداشتن، نگاشت محیط، نقشه برداری معمولی، سایه های بلادرنگ، و انسداد محیط.
    }

        {  \large
    برای مبتدیان، بهتر است این کتاب از ابتدا به انتها خوانده شود. فصل ها به گونه ای سازماندهی شده اند که با هر فصل، میزان سختی به تدریج افزایش می یابد. به این ترتیب، هیچ جهش ناگهانی در پیچیدگی وجود ندارد که خواننده را سردرگم کند.
    به طور کلی، برای یک فصل خاص، از تکنیک ها و مفاهیمی که قبلاً توسعه داده شده است استفاده خواهیم کرد. بنابراین، مهم است که قبل از ادامه، بر مطالب یک فصل تسلط داشته باشید.
    خوانندگان با تجربه می توانند فصل های مورد علاقه خود را انتخاب کنند. در نهایت، ممکن است از خود بپرسید که پس از خواندن این کتاب چه نوع بازی هایی را می توانید توسعه دهید. پاسخ این سوال را بهتر است با مرور این کتاب و مشاهده انواع برنامه های توسعه یافته به دست آورید. از این رو باید بتوانید انواع بازی هایی را که می توان بر اساس تکنیک های آموزش داده شده در این کتاب و برخی از نبوغ خود توسعه داد، تجسم کنید.
    }
}
%--------------------------------------%
\newpage

\title{
    \LARGE
    \textbf{مخاطبان این کتاب}
}
\\ \rule{\textwidth}{0.5pt}
{  \large
این کتاب با در نظر گرفتن سه مخاطب زیر طراحی شده است:
    \begin{enumerate}
        \item {برنامه نویسان سطح متوسط \lr{C++} که می خواهند مقدمه ای بر برنامه نویسی سه بعدی با استفاده از آخرین نسخه \lr{Direct3D}}
        \item {برنامه نویسان سه بعدی با یک API غیر از DirectX (به عنوان مثال، \lr{OpenGL}) تجربه کرده اند و می خواهند یک مقدمه برای \lr{Direct3D 12} داشته باشند.}
        \item {برنامه نویسان با تجربه \lr{Direct3D} که مایل به یادگیری آخرین نسخه \lr{Direct3D} هستند}
    \end{enumerate}
}
\\[25pt]

\title{
    \LARGE
    \textbf{پیش نیاز ها}
}
\\ \rule{\textwidth}{0.5pt}
{  \large
لازم به ذکر است که این مقدمه ای بر \lr{Direct3D 12}، برنامه نویسی \lr{shader} و برنامه نویسی بازی های سه بعدی است. این مقدمه ای برای برنامه نویسی کامپیوتر به صورت کلی نیست. خواننده باید شرایط زیر را رعایت کند:
    \begin{enumerate}
        \item {ریاضیات دبیرستان: مثلاً جبر، مثلثات و توابع (ریاضی).}
        \item {تجربه با \lr{Visual Studio}: باید بداند که چگونه پروژه ها را ایجاد کند، فایل ها را اضافه کند و کتابخانه های خارجی را \lr{link} کند.}
        \item {\lr{C++} متوسط و مهارت های ساختار داده: به عنوان مثال، با اشاره گرها، آرایه ها، \lr{overload} اپراتورها، لیست های پیوندی، وراثت و چندشکلی مشکلی ندارد.}
        \item {آشنایی با برنامه نویسی ویندوز با \lr{Win32 API} مفید است، اما لازم نیست. ما یک پرایمر \lr{Win32} را در ضمیمه A ارائه می دهیم.}
    \end{enumerate}
}

\\[25pt]
\title{
    \LARGE
    \textbf{ابزارها و سخت افزارهای توسعه مورد نیاز}
}
\\ \rule{\textwidth}{0.5pt}
{  \large
برای برنامه نویسی برنامه های \lr{Direct3D 12} به موارد زیر نیاز است:
    \begin{enumerate}
        \item {\lr{Windows 10}.}
        \item {\lr{Visual Studio 2015} یا جدیدتر.}
        \item {کارت گرافیکی که از \lr{Direct3D 12} پشتیبانی می کند. دموهای این کتاب روی \lr{Geforce GTX 760} تست شده اند.}
    \end{enumerate}
}


%--------------------------------------%
\newpage