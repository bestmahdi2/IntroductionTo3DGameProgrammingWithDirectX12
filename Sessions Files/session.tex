
\def \Session {1}
\setcounter{chapter}{\Session-1}

\part{پیشنیاز های ریاضی}

{
بازی های ویدیویی سعی در شبیه سازی دنیای مجازی دارند.
با این حال، کامپیوترها، به دلیل ماهیت خود، اعداد را محاسبه می کنند. بنابراین مشکل چگونگی انتقال یک جهان به یک کامپیوتر مطرح می شود.
پاسخ میتواند اینگونه باشد که جهان‌های ما و فعل و انفعالات موجود در آن را کاملاً ریاضی توصیف کنیم. در نتیجه، ریاضیات نقش اساسی در توسعه بازی های ویدیویی ایفا می کند.

در این بخش، ابزارهای ریاضی را که در سراسر این کتاب مورد استفاده قرار قرار گرفته اند، معرفی می کنیم. تأکید ما بر بردارها، سیستم های مختصات، ماتریس ها و تبدیل ها است، زیرا این ابزارها تقریباً در همه ی برنامه های نمونه این کتاب استفاده شده اند.
علاوه بر توضیحات ریاضی، بررسی و نمایش کلاس ها و توابع مربوطه از کتابخانه ریاضی DirectX ارائه شده است.

توجه داشته باشید که موضوعاتی که در اینجا مورد بررسی قرار می‌گیرند، تنها مواردی هستند که برای درک ادامه ی این کتاب ضروری هستند.
این کتاب به هیچ وجه راه حل جامع ریاضیات بازی های ویدیویی نیست.
برای خوانندگانی که مایل به ارجاع کامل تر به ریاضیات بازی های ویدیویی هستند، [Verth04] و [Lengyel02] را توصیه می کنیم.
}
{
فصل 1 جبر برداری: بردارها اساسی ترین اشیاء ریاضی مورد استفاده در بازی های کامپیوتری هستند.
ما از بردارها برای نشان دادن موقعیت ها، جابجایی ها، جهت ها، سرعت ها و نیروها استفاده می کنیم.
در این فصل، بردارها و عملیات مورد استفاده برای کار با آنها را مطالعه می کنیم.

فصل 2، جبر ماتریسی: ماتریس ها روشی کارآمد و فشرده برای نمایش تبدیل ها ارائه می دهند.
در این فصل با ماتریس ها و عملیات تعریف شده بر روی آنها آشنا می شویم.

فصل 3، تبدیل: این فصل سه تبدیل هندسی اساسی را بررسی می کند: مقیاس بندی، چرخش و ترجمه.
ما از این تبدیل ها برای کار با اشیاء سه بعدی در فضا استفاده می کنیم.
علاوه بر این، تغییر تبدیل مختصات را توضیح می دهیم، که برای تبدیل مختصات نشان دهنده هندسه از یک سیستم مختصات به سیستم دیگر استفاده می شود.
}
%--------------------------------------%

\chapter{جبر برداری}

{
بردارها نقش مهمی در گرافیک کامپیوتری، تشخیص برخورد و شبیه سازی فیزیکی ایفا می کنند که همگی اجزای رایج در بازی های ویدئویی مدرن هستند.
رویکرد ما در اینجا غیر رسمی و عملی است به همین دلیل پیشنهاد ما [Verth04] است که کتابی اختصاصی برای ریاضیات بازی های سه بعدی/گرافیک است.
ما بر اهمیت بردارها بسیار تأکید داریم زیرا تقریباً در همه ی برنامه های آزمایشی در این کتاب استفاده شده اند.

اهداف:

1. یادگیری نحوه نمایش بردارها به صورت هندسی و عددی
2. پی بردن به عملیات تعریف شده بر روی بردارها و کاربردهای هندسی آنها
3. آشنایی با توابع برداری و کلاس های کتابخانه DirectXMath
}

{
بردار ها:

بردار به کمیتی اشاره دارد که هم مقدار و هم جهت دارد.
به کمیت هایی که هم مقدار و هم جهت دارند، کمیت های بردار می گویند.
نمونه‌هایی از کمیت‌های با ارزش برداری عبارتند از نیروها (نیروی در جهت خاصی با قدرت - قدر معین اعمال می‌شود)، جابه‌جایی (جهت خالص و فاصله حرکت ذره)، و سرعت‌ها (سرعت و جهت).
بنابراین، بردارها برای نمایش نیروها، جابجایی ها و سرعت ها استفاده می شوند.
علاوه بر این، ما همچنین از بردارها برای تعیین جهات خالص استفاده می کنیم، مانند جهتی که بازیکن در یک بازی سه بعدی به دنبال آن است،
جهتی که یک چند ضلعی رو به آن قرار دارد، جهتی که پرتوی نور در آن حرکت می کند،
یا جهتی که پرتوی در آن حرکت می کند. نور از یک سطح منعکس می شود.

اولین مرحله در توصیف ریاضی بردار از نظر هندسی است: ما به صورت گرافیکی یک بردار را با یک پاره خط جهت دار مشخص می کنیم (شکل 1.1 را ببینید)، که در آن طول نشان دهنده بزرگی بردار و هدف نشان دهنده جهت بردار است.
توجه می‌کنیم که مکانی که در آن یک بردار رسم می‌کنیم بی‌اهمیت است، زیرا تغییر مکان، بزرگی یا جهت را تغییر نمی‌دهد (دو ویژگی که یک بردار دارد).
بنابراین می گوییم دو بردار اگر و فقط در صورتی مساوی هستند که طول یکسانی داشته باشند و در یک جهت باشند.
بنابراین، بردارهای u و v ترسیم شده در شکل 1.1a در واقع برابر هستند زیرا طول و نقطه یکسانی دارند.
در واقع، چون مکان برای بردارها اهمیتی ندارد، ما همیشه می‌توانیم یک بردار را بدون تغییر معنای آن ترجمه کنیم (زیرا ترجمه نه طول و نه جهت را تغییر می‌دهد).
توجه داشته باشید که ما می‌توانیم u را طوری ترجمه کنیم که کاملاً با v همپوشانی داشته باشد (و بالعکس) و در نتیجه آنها را غیرقابل تشخیص کنیم - بنابراین برابری آنها.

به عنوان یک مثال فیزیکی، بردارهای u و v در شکل 1.1b هر دو به مورچه ها در دو نقطه مختلف A و B می گویند که ده متر از جایی که هستند به سمت شمال حرکت کنند.
دوباره آن u=v را داریم. بردارها خود مستقل از موقعیت هستند.
آنها به سادگی به مورچه ها آموزش می دهند که چگونه از جایی که هستند حرکت کنند.
در این مثال، آنها به مورچه ها می گویند که ده متر (طول) به سمت شمال (جهت) حرکت کنند.
}
{
بردارها و سیستم های مختصات:

اکنون می‌توانیم عملیات هندسی مفیدی را روی بردارها تعریف کنیم، که سپس می‌توان از آنها برای حل مسائل مربوط به کمیت‌های بردار استفاده کرد.
با این حال، از آنجایی که کامپیوتر نمی تواند با بردارها به صورت هندسی کار کند، باید راهی برای تعیین عددی بردارها پیدا کنیم.
بنابراین کاری که ما انجام می دهیم این است که یک سیستم مختصات سه بعدی را در فضا معرفی می کنیم و همه بردارها را طوری ترجمه می کنیم که دنباله آنها با مبدا منطبق باشد (شکل 1.2).
سپس می‌توانیم یک بردار را با تعیین مختصات سر آن شناسایی کنیم و مانند شکل 1.3، v= (x, y, z) را بنویسیم.
اکنون می توانیم یک بردار را با سه شناور در یک برنامه کامپیوتری نشان دهیم

اگر به صورت دو بعدی کار می کنیم، فقط از یک سیستم مختصات دو بعدی استفاده می کنیم و بردار فقط دو مختصات دارد: v= (x، y) و می توانیم یک بردار را با دو شناور در یک برنامه کامپیوتری نشان دهیم.

شکل 1.4 را در نظر بگیرید که یک بردار v و دو فریم در فضا را نشان می دهد. (توجه داشته باشید که ما از اصطلاحات چارچوب، چارچوب مرجع، فضا و سیستم مختصات استفاده می کنیم تا همگی در این کتاب به یک معنا باشند.)
می توانیم v را طوری ترجمه کنیم که در هر یک از دو فریم در موقعیت استاندارد قرار گیرد. با این حال، مشاهده کنید که مختصات بردار v نسبت به قاب A با مختصات بردار v نسبت به قاب B متفاوت است.
به عبارت دیگر، همان بردار v نمایش مختصاتی متفاوتی برای فریم های متمایز دارد.

این ایده شبیه به مثلا دما است. آب در 100 درجه سانتیگراد یا 212 درجه فارنهایت می جوشد.
دمای فیزیکی آب جوش بدون توجه به مقیاس یکسان است (یعنی نمی‌توانیم نقطه جوش را با انتخاب مقیاس متفاوت کاهش دهیم)، اما بر اساس مقیاسی که استفاده می‌کنیم عدد اسکالر متفاوتی را به دما اختصاص می‌دهیم.
به طور مشابه، برای یک بردار، جهت و بزرگی آن، که در پاره خط جهت دار تعبیه شده است، تغییر نمی کند.
فقط مختصات آن بر اساس چارچوب مرجعی که برای توصیف آن استفاده می کنیم تغییر می کند.
این مهم است زیرا به این معنی است که هرگاه یک بردار را با مختصات شناسایی کنیم، آن مختصات نسبت به برخی از چارچوب های مرجع هستند.
اغلب در گرافیک های کامپیوتری سه بعدی، از بیش از یک فریم مرجع استفاده می کنیم و بنابراین، باید ردیابی کنیم که مختصات یک بردار نسبت به کدام فریم است.
علاوه بر این، ما باید بدانیم که چگونه مختصات برداری را از یک فریم به فریم دیگر تبدیل کنیم.

می بینیم که هر دو بردار و نقاط را می توان با مختصات (x, y, z) نسبت به یک قاب توصیف کرد.
با این حال، آنها یکسان نیستند; یک نقطه نشان دهنده یک مکان در 3 فاصله است، در حالی که یک بردار نشان دهنده یک قدر و جهت است.
در مورد نکات در §1.5 بیشتر برای گفتن خواهیم داشت.
}

%--------------------------------------%
\rule{\textwidth}{0.5pt}

\def \Subject {جبر برداری}
\def \Author {علی بدیعی}
\def \Date {1402/4/22}
\def \Session {2}
\setcounter{chapter}{\Session-1}

\chapter{\Subject}
\chapterauthor{\Author~ - \Date}

%--------------------------------------%
\rule{\textwidth}{0.5pt}

\def \Subject {انتقال ها}
\def \Author {علی بدیعی}
\def \Date {1402/4/22}
\def \Session {3}
\setcounter{chapter}{\Session-1}

\chapter{\Subject}
\chapterauthor{\Author~ - \Date}

%--------------------------------------%
\rule{\textwidth}{0.5pt}

جزوه جلسه
\Session ام
 مورخ 
\Date
درس 
\CourseName
 تهیه شده توسط
 \Author. 


 
\section{معیار ارزیابی جزوه}
معیارهای مورد استفاده برای ارزشیابی کیفیت جزوه به شرح زیر است:
\begin{itemize}
    \item    
     پوشش کامل مطالب
    \item    
     رعایت قواعد نگارشی دستور زبان فارسی
    \item    
     استفاده از اشکال مناسب
    \item    
     اشاره به منابع کمک آموزشی
\end{itemize}


برای نوشتن شبه‌کد
 \footnote{pseudocode}
 می‌توانید از مثال زیر در الگوریتم
  \ref{pseudocode1}
   استفاده کنید :

\begin{latin}
\begin{algorithm}[H]
    \KwData{this text}
    \KwResult{how to write algorithm with \LaTeX2e }
    initialization\;
    \While{not at end of this document}{
        read current\;
        \eIf{understand}{
            go to next section\;
            current section becomes this one\;
        }{
            go back to the beginning of current section\;
        }
    }
    \label{pseudocode1}
    \caption{How to write algorithms}
\end{algorithm}
\end{latin}

همچنین برای اضافه کردن شکل می‌توانید از شکل زیر استفاده کنید و برای ارجاع دادن به بصورت شکل
 \ref{fig:dynamicprogramming}
 استفاده کنید.
 همچنین برای درج کلمات انگلیسی در پاراگراف فارسی می‌توان به این شکل
 \lr{English Text}
 یا برای تاکید به این شکل
 \grayBox{English Text}
 عمل کرد.
\begin{figure}[H]
    \centering
    \includegraphics[width=0.4\linewidth]{images/SKU_logo_color.jpg}
    \caption{برنامه نویسی پویا}
    \label{fig:dynamicprogramming}
\end{figure}

برای گنجاندن قطعه‌ای از کد به زبان‌های 
\grayBox{Java, C\#, C++, Python} 
از فایل اصلی کد می‌توانید به شکل زیر عمل کنید.
 
\begin{program}[H]
    \inputminted[frame=none]{csharp}{Sample.cs}
    \caption{تابع فیبوناچی در سی‌شارپ}
    \label{csharpfib}
\end{program}

به عنوان مثال در 
\programref{csharpfib}
پیاده‌سازی تابع فیبوناچی به زبان سی‌شارپ را ملاحظه می‌کنید. این تابع ورودی $n$ را دریافت کرده و عدد $n$ام دنباله را برمی‌گرداند. همانطور که ملاحظه می‌کنید این تابع به صورت بازگشتی پیاده‌سازی شده است.

\begin{program}[H]
\inputminted[frame=none]{cpp}{Sample.cpp}
\caption{تابع فیبوناچی در \grayBox{C++} }
\label{cppfib}
\end{program}

با مقایسه متن لاتک و خروجی به این مساله پی‌ خواهید برد که برای اشاره به یک قطعه کد در لاتک لازم است از دستور
\grayBox{label}
داخل محیط/بخش 
\grayBox{program}
استفاده کرده و برای اشاره به آن بخش از کد از دستور
\grayBox{programref}
استفاده نمایید. لازم است پارامتر داده شده به هر دو دستور یکی باشد تا بدرستی به قطعه کد مورد نظر اشاره کنید(\programref{pythonfib}). 

\begin{program}[H]
\inputminted[frame=none]{python}{sample.py}
\caption{تابع فیبوناچی در پایتون }
\label{pythonfib}
\end{program}

لطفا در همه موارد به جزئیات رعایت شده در متن لاتک دقت کنید. مخصوصا برای راحتی ویرایش متن لاتک بهتر است دستورات لاتین لاتک در خطوط جداگانه از متن فارسی قرار داده شود (\programref{javafib}).

\begin{program}[H]
\inputminted[frame=none]{java}{Sample.java}
\caption{تابع فیبوناچی در جاوا }
\label{javafib}
\end{program}
 چنانچه بخواهید قطعه کد را مستقیم در متن جای دهید، می‌توانید از الگوی زیر پیروی کنید (\programref{csharpnode}).
 
\begin{program}[H]
\begin{minted}[frame=none]{csharp}
public class Node<_Type>
{
    public _Type Value;
    public Node<_Type> Next;
}
\end{minted}
\caption{تعریف لیست پیوندی در سی‌شارپ}
\label{csharpnode}
\end{program}
 
همانطور که در نمونه‌کدها ملاحظه می‌کنید فاصله‌ها دقیقا همانطور که در لاتک آمده است در خروجی نمایش داده می‌شود.
 
\begin{program}[H]
\begin{minted}[frame=none]{java}
public class Node<_Type> {
    public _Type Value;
    public Node<_Type> Next;
}
\end{minted}
\label{javanode}
\caption{تعریف لیست پیوندی در ‌جاوا}
\end{program}

لذا لازم است که فاصله‌های ابتدای خط را مرتب و منظم و بدون کوچکترین اشکالی رعایت کنید تا خروجی مستند شما نیز بدون اشکال باشد.

\begin{program}[H]
\begin{minted}[frame=none]{cpp}
template<typename _Type>
class Node
{
public:
    _Type Value;
    Node<_Type>* Next;
}
\end{minted}
\label{cppnode}
\caption{تعریف لیست پیوندی در ‌\grayBox{C++}}
\end{program}
در نهایت استفاده از دستور 
\grayBox{grayBox}
نیز می‌تواند کمک شایانی به زیبایی و خوانایی متن شما بکند. این دستور علاوه بر عوض کردن رنگ پس‌زمینه از فونت انگلیسی با عرض ثابت نیز استفاده می‌کند که برای کلمات کلیدی یا اسامی متغیرها در کد قابل استفاده است.

\begin{program}[H]
\begin{minted}[frame=none]{python}
class Node: 
def __init__(self, data): 
    self.value = data
    self.next = None  
\end{minted}
\label{pythonnode}
\caption{تعریف لیست پیوندی در ‌پایتون}
\end{program}

\grayLBox{
    AAC\\
    ACG\\
    GAA\\
    GTT\\
    TCG\\
}
\\
همچنین سعی کنید حتی‌الامکان به منابع و مراحع مناسب ارجاع دهید
\cite{CLRS}. 
علاوه بر مراجع چنانچه ابزار یا وب‌سایت قابل توجهی موجود است خوب است به آن هم ارجاع دهید
\cite{visualizationwebsite}. 

