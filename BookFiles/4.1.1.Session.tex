%-----------------------------------------------------------------------------------------------------------%
\newpage

\def \Session {1}
\setcounter{chapter}{\Session-1}

\part{پیشنیاز های ریاضی}

{
    \Large
    \begin{spacing}{1.5}
        بازی های ویدیویی سعی در شبیه سازی دنیای مجازی دارند.
        با این حال، کامپیوترها، به دلیل ماهیت خود، اعداد را محاسبه می کنند. بنابراین مشکل چگونگی انتقال یک جهان به یک کامپیوتر مطرح می شود.
        پاسخ این مشکل میتواند اینگونه باشد که جهان‌های ما و فعل و انفعالات موجود در آن را کاملاً ریاضی توصیف کنیم.
        در نتیجه، ریاضیات نقش اساسی در توسعه بازی های ویدیویی ایفا می کند.

        در این بخش، ابزارهای ریاضی که در این کتاب مورد استفاده قرار گرفته اند، معرفی می کنیم. تأکید ما بر بردارها، سیستم های مختصات، ماتریس ها و تبدیل ها است، زیرا این ابزارها تقریباً در همه ی برنامه های نمونه این کتاب استفاده شده اند.
        علاوه بر توضیحات ریاضی، بررسی و نمایش کلاس ها و توابع مربوطه از کتابخانه ریاضی DirectX را نیز ارائه میکنیم.

        توجه داشته باشید که موضوعاتی که در اینجا مورد بررسی قرار می‌گیرند، تنها مواردی هستند که برای درک ادامه ی این کتاب ضروری اند.
        این کتاب به هیچ وجه راه حل جامعی برای ریاضیات بازی های ویدیویی نیست.

    \end{spacing}

    \begin{theo}{thm:pythagoras}
        \Large
        برای خوانندگانی که مایل به ارجاع کامل تر به ریاضیات بازی های ویدیویی هستند، کتاب های زیر را توصیه می کنیم.
        \lr{
            \begin{enumerate}
                \item {Essential Mathematics for Games and Interactive Applications: A Programmer's Guide (Verth04)}
                \item {Mathematics for 3D Game Programming and Computer Graphics (Lengyel02)}
            \end{enumerate}
        }
    \end{theo}
}
\newpage
{
    \Large
    \begin{spacing}{1.5}
        \textbf{فصل 1، جبر برداری:} بردارها اساسی ترین اشیاء ریاضی مورد استفاده در بازی های کامپیوتری هستند.
        ما از بردارها برای نشان دادن موقعیت ها، جابجایی ها، جهت ها، سرعت ها و نیروها استفاده می کنیم.
        در این فصل، بردارها و عملیات مورد استفاده برای کار با آنها را مطالعه می کنیم.

        \textbf{فصل 2، جبر ماتریسی:} ماتریس ها روشی کارآمد و فشرده برای نمایش تبدیل ها ارائه می دهند.
        در این فصل با ماتریس ها و عملیات تعریف شده بر روی آنها آشنا می شویم.

        \textbf{فصل 3، تبدیل:} این فصل سه تبدیل هندسی اساسی را بررسی می کند: مقیاس بندی، چرخش و انتقال.
        ما از این تبدیل ها برای کار با اشیاء سه بعدی در فضا استفاده می کنیم.
        علاوه بر این، تغییر تبدیل مختصات را توضیح می دهیم، که برای تبدیل مختصات هندسی از یک سیستم مختصات به سیستم دیگر استفاده می شود.
    \end{spacing}
}

%-----------------------------------------------------------------------------------------------------------%
\newpage

\chapter{جبر برداری}

{
    \Large
    \begin{spacing}{1.5}
        بردارها نقش مهمی در گرافیک کامپیوتری، تشخیص برخورد و شبیه سازی فیزیکی ایفا می کنند که همگی اجزای رایج در بازی های ویدئویی مدرن هستند.
        رویکرد ما در اینجا غیر تخصصی و عملی است به همین دلیل پیشنهاد ما کتاب \lr{Verth04} (در نکته ی قبل ذکر شده است) است که کتابی اختصاصی برای ریاضیات بازی های سه بعدی/گرافیک است.
        ما بر اهمیت بردارها بسیار تأکید داریم زیرا در بیشتر برنامه های آزمایشی این کتاب استفاده شده اند.

        \textbf{اهداف:}

        \begin{enumerate}
            \item {یادگیری نحوه نمایش بردارها به صورت هندسی و عددی}
            \item {یادگیری عملیات تعریف شده بر روی بردارها و کاربردهای هندسی آنها}
            \item {آشنایی با توابع برداری و کلاس های کتابخانه \lr{DirectXMath}}
        \end{enumerate}
    \end{spacing}
}
\newpage
{
    \textbf{\BTitr بردار ها:}

    \Large
    \begin{spacing}{1.5}
        بردار به کمیتی اشاره دارد که هم اندازه و هم جهت دارد.
        به کمیت هایی که هم اندازه و هم جهت دارند، کمیت های بردار می گویند.
        نمونه‌هایی از کمیت‌های برداری عبارتند از نیروها (نیرو در جهت خاصی با قدرت/اندازه معین اعمال می‌شود)، جابه‌جایی (جهت برآیند و فاصله حرکت ذره)، و سرعت‌ها (سرعت و جهت).
        بنابراین، بردارها برای نمایش نیروها، جابجایی ها و سرعت ها استفاده می شوند.
        علاوه بر این، ما همچنین از بردارها برای تعیین جهات خالص استفاده می کنیم،
        مانند جهتی که بازیکن در یک بازی سه بعدی به آن نگاه میکند،
        جهتی که یک چند ضلعی رو به آن قرار دارد، جهتی که پرتوی نور در آن حرکت می کند،
        یا جهتی که در آن یک پرتو نور از یک سطح منعکس می شود.

        اولین مرحله در توصیف ریاضی بردار ، توصیف هندسی آن است: ما به صورت گرافیکی یک بردار را با یک پاره خط جهت دار مشخص می کنیم (شکل \ref{fig:4.Session.1.1.1})، که در آن طول نشان دهنده بزرگی بردار و سر کمان نشان دهنده جهت بردار است.
        به این نکته باید توجه داشته باشیم که مکانی که در آن یک بردار رسم می‌کنیم بی‌اهمیت است، زیرا تغییر مکان، بزرگی یا جهت را تغییر نمی‌دهد (دو ویژگی که یک بردار دارد).
        بنابراین می گوییم دو بردار برابر هستند اگر و تنها اگر طول یکسانی داشته باشند و در یک جهت باشند.
        بنابراین، بردارهای \lr{u} و \lr{v} ترسیم شده در قسمت آ شکل \ref{fig:4.Session.1.1.1} در واقع برابر هستند زیرا طول و جهت یکسانی دارند.
        در واقع، چون مکان برای بردارها اهمیتی ندارد، ما همیشه می‌توانیم یک بردار را بدون تغییر هویت آن ، انتقال دهیم (زیرا انتقال نه طول و نه جهت را تغییر می‌دهد).
        توجه داشته باشید که ما می‌توانیم \lr{u} را طوری انتقال دهیم که کاملاً بر \lr{v} همپوشانی داشته باشد (و بالعکس) و در نتیجه آنها را غیرقابل تشخیص کنیم. این نیز دلیلی برای برابری آنهاست.

        \begin{figure}[H]
            \centering
            \setlength{\belowcaptionskip}{-10pt}
            \includegraphics[width=\textwidth]{Images/4/4.Session.1.1.1}
            \caption{(الف) بردار هایی که روی صفحه ی دوبعدی کشیده شده اند.
                (ب) بردار هایی که مورچه ها را برای حرکت 10 متر در جهت شمال راهنمایی میکنند.}
            \label{fig:4.Session.1.1.1}
        \end{figure}

        به عنوان یک مثال فیزیکی، بردارهای \lr{u} و \lr{v} در قسمت ب شکل \ref{fig:4.Session.1.1.1} هر دو به مورچه ها میگویند در دو نقطه مختلف \lr{A} و \lr{B} ده متر به سمت شمال حرکت کنند.
        دوباره \lr{u = v} را داریم.
        بردارها خود مستقل از موقعیت هستند و
        آنها به سادگی به مورچه ها آموزش می دهند که چگونه از جایی که هستند، ده متر (طول) به سمت شمال (جهت) حرکت کنند.
    \end{spacing}
}

{
    \textbf{\BTitr بردارها و سیستم های مختصات:}

    \Large
    \begin{spacing}{1.5}
        اکنون می‌توانیم عملیات هندسی مفیدی را روی بردارها تعریف کنیم، که سپس می‌توان از آنها برای حل مسائل مربوط به کمیت‌های برداری استفاده کرد.
        با این حال، از آنجایی که کامپیوتر نمی تواند با بردارها به صورت هندسی کار کند، باید راهی برای تعیین عددی بردارها پیدا کنیم.
        بنابراین یک سیستم مختصات سه بعدی را در فضا معرفی می کنیم و همه بردارها را طوری انتقال میدهیم که دم آن ها با مبدا منطبق باشد (شکل \ref{fig:4.Session.1.1.2}).
        سپس می‌توانیم یک بردار را با تعیین مختصات سر آن شناسایی کنیم و مانند شکل \ref{fig:4.Session.1.1.3}، \lr{v = (x, y, z)} را بنویسیم.
        اکنون می توانیم یک بردار را با سه \lr{float} در یک برنامه کامپیوتری نشان دهیم.

        \begin{theo}{thm:pythagoras}
            \Large
            اگر به صورت دو بعدی کار کنیم، فقط از یک سیستم مختصات دو بعدی استفاده می کنیم و بردار فقط دو مختصات دارد:
            \lr{v = (x, y)} و می توانیم یک بردار را با دو \lr{float} در یک برنامه کامپیوتری نشان دهیم.
        \end{theo}

        \begin{figure}[H]
            \centering
            \setlength{\belowcaptionskip}{-10pt}
            \includegraphics[width=0.5\textwidth]{Images/4/4.Session.1.1.2}
            \caption{\lr{v} را طوری انتقال میدهیم که دم آن با مبدأ سیستم مختصات منطبق
            باشد. وقتی دم یک بردار با مبدا منطبق می شود، می گوییم که در موقعیت استاندارد قرار دارد.}
            \label{fig:4.Session.1.1.2}
        \end{figure}

        \begin{figure}[H]
            \centering
            \setlength{\belowcaptionskip}{-10pt}
            \includegraphics[width=0.5\textwidth]{Images/4/4.Session.1.1.3}
            \caption{یک بردار مشخص شده توسط مختصات نسبت به یک سیستم مختصات.}
            \label{fig:4.Session.1.1.3}
        \end{figure}

        شکل \ref{fig:4.Session.1.1.4} را در نظر بگیرید که یک بردار \lr{v} و دو فریم (\lr{frame}) در فضا را نشان می دهد. (توجه داشته باشید که ما از اصطلاحات فریم، فریم مرجع، فضا و سیستم مختصات استفاده می کنیم که همگی در این کتاب به یک معنا هستند.)
        می توانیم \lr{v} را طوری انتقال دهیم که در هر یک از دو سیستم مختصات در موقعیت استاندارد قرار گیرد. با این حال، مشاهده کنید که مختصات بردار \lr{v} نسبت به سیستم مختصات \lr{A} با مختصات بردار \lr{v} نسبت به سیستم مختصات \lr{B} متفاوت است.
        به عبارت دیگر، همان بردار \lr{v} نمایش مختصاتی متفاوتی برای سیستم مختصات های متمایز دارد.

        \begin{figure}[H]
            \centering
            \setlength{\belowcaptionskip}{-10pt}
            \includegraphics[scale=0.7]{Images/4/4.Session.1.1.4}
            \caption{همان بردار v زمانی که نسبت به سیستم مختصات های مختلف تعریف شود، مختصات متفاوتی دارد.}
            \label{fig:4.Session.1.1.4}
        \end{figure}

        این ایده شبیه به مثال دما است. آب در 100 درجه سانتیگراد یا 212 درجه فارنهایت می جوشد.
        دمای فیزیکی آب جوش بدون توجه به مقیاس یکسان است (یعنی نمی‌توانیم نقطه جوش را با انتخاب مقیاس متفاوت کاهش دهیم)،
        اما بر اساس مقیاسی که استفاده می‌کنیم عدد اسکالر متفاوتی را به دما اختصاص می‌دهیم.
        به طور مشابه، برای یک بردار، جهت و بزرگی آن، که در پاره خط جهت دار تعبیه شده است، تغییر نمی کند.
        فقط مختصات آن بر اساس سیستم مختصات که برای توصیف آن استفاده می کنیم تغییر می کند.
        این نکته ی مهمی ست زیرا به این معنی است که هرگاه یک بردار را با مختصات شناسایی کنیم، آن مختصات نسبت به برخی از سیستم های مختصات هستند.
        اغلب در گرافیک های کامپیوتری سه بعدی، از بیش از یک سیستم مختصات استفاده می کنیم و بنابراین، باید شناسایی کنیم که مختصات یک بردار نسبت به کدام سیستم مختصات است.
        علاوه بر این، ما باید بدانیم که چگونه مختصات برداری را از یک سیستم مختصات به سیستم مختصات دیگر تبدیل کنیم.

        \begin{theo}{thm:pythagoras}
            \Large
            مشاهده میکنیم که بردارها و نقاط را می توان با مختصات (\lr{x, y, z}) نسبت به یک سیستم مختصات توصیف کرد.
            با این حال، بردارها و نقاط یکسان نیستند؛ یک نقطه نشان دهنده یک مکان در 3 فاصله است، در حالی که یک بردار نشان دهنده یک اندازه و جهت است.
        \end{theo}
    \end{spacing}
}

{
    \textbf{سیستم\BTitr  های مختصات چپگرد در مقابل راستگرد}

    \Large
    \begin{spacing}{1.5}
        \lr{Direct3D} از یک سیستم مختصات چپگرد استفاده می کند.
        اگر دست چپ خود را به صورتی بگیرید که انگشتان خود را به سمت محور x مثبت گرفته و سپس انگشتان خود را به سمت محور y مثبت خم کنید، انگشت شست شما در جهت محور z مثبت قرار می گیرد.
        شکل \ref{fig:4.Session.1.1.5} تفاوت بین یک سیستم مختصات چپ دست و راست دست را نشان می دهد.
        توجه کنید که در سیستم مختصات راستگرد، اگر دست راست خود را به صورتی بگیرید که انگشتان خود را به سمت محور x مثبت گرفته و سپس انگشتان خود را به سمت محور y مثبت خم کنید، انگشت شست شما در جهت محور z مثبت اشاره می کند.

        \begin{figure}[H]
            \centering
            \setlength{\belowcaptionskip}{-10pt}
            \includegraphics[scale=0.7]{Images/4/4.Session.1.1.5}
            \caption{در سمت چپ ما یک سیستم مختصات چپگرد داریم که محور z مثبت وارد صفحه می شود. در سمت راست ما یک سیستم مختصات راستگرد داریم که محور z مثبت از صفحه خارج می شود.}
            \label{fig:4.Session.1.1.5}
        \end{figure}
    \end{spacing}
}

{
    \textbf{\BTitr عملیات بردار پایه}

    \Large
    \begin{spacing}{1.5}
        اکنون با استفاده از نمایش مختصاتی، تساوی، جمع، ضرب اسکالر و تفریق را بر روی بردارها تعریف می کنیم.
        برای این چهار تعریف، فرض میکنیم $\textbf{u}=(u_{x},u_{y},u_{z})$ و  $\textbf{v}=(v_{x},v_{y},v_{z})$.

        \begin{enumerate}
            \item {دو بردار مساوی هستند اگر و تنها اگر اجزای متناظر آنها با هم برابر باشند.
            یعنی $\textbf{u}=\textbf{v}$ اگر و تنها اگر $u_{x}=v_{x}$ ، $u_{y}=v_{y}$ و $u_{z}=v_{z}$}
            \item {بردارها را به صورت جزء اضافه می کنیم: $\textbf{u}+\textbf{v}=(u_{x}+v_{x},u_{y}+v_{y},u_{z}+v_{z})$.
            توجه داشته باشید که فقط اضافه کردن بردارهایی با همان بعد ، منطقی ست.}
            \item {می توانیم یک اسکالر (یعنی یک عدد حقیقی) و یک بردار را ضرب کنیم و نتیجه یک بردار خواهد بود.
            فرض کنید k یک اسکالر باشد، پس $k\textbf{u}=(ku_{x},ku_{y},ku_{z})$. به این ضرب اسکالر می گویند.}
            \item {ما تفریق را بر حسب جمع بردار و ضرب اسکالر انجام می دهیم.
            \\یعنی $\textbf{u}-\textbf{v}=\textbf{u}+(-1\cdot\textbf{v})=\textbf{u}+(-\textbf{v})=(u_{x}-v_{x},u_{y}-v_{y},u_{z}-v_{z})$}
        \end{enumerate}

        مثال***
        فرض کنید $\textbf{u}=(1,2,3), \textbf{v}=(1,2,3), \textbf{w}=(3,0,-2), k=2$
        \lr{
            \begin{enumerate}
                \item {$\textbf{u}+\textbf{w}=(1,2,3)+(3,0,-2)=(4,2,1)$;}
                \item {$\textbf{u}=\textbf{v}$}
                \item {$\textbf{u}-\textbf{v}=\textbf{u}+(-\textbf{v})=(1,2,3)+(-1,-2,-3)=(0,0,0)=\textbf{0}$;}
                \item {$k\textbf{w}=2(3,0,-2)=(6,0,-4)$}
            \end{enumerate}
        }
        تفاوتی که در مورد سوم هست ، یک بردار خاص به نام بردار صفر را نشان می دهد که همه اجزای آن صفر است و با \lr{\textbf{0}} نشان داده می شود.


        مثال***
        ما این مثال را با بردارهای دوبعدی برای ساده‌تر کردن کار توضیح می‌دهیم. ایده ها مانند فضای سه بعدی هستند، فقط با یک جزء کمتر ، به صورت دو بعدی کار می کنیم.

        \begin{enumerate}
            \item {فرض کنید $\textbf{v}=(2,1)$، $\textbf{v}$ و $-\frac{1}{2}\textbf{v}$ چگونه از نظر هندسی با هم مقایسه می شوند؟
            توجه داریم که $-\frac{1}{2}\textbf{v}=(-1,-\frac{1}{2})$.
            با ترسیم نمودار $\textbf{v}$ و $-\frac{1}{2}\textbf{v}$ (قسمت آ شکل \ref{fig:4.Session.1.1.6})،
            متوجه می‌شویم که $-\frac{1}{2}\textbf{v}$ در جهت مخالف $\textbf{v}$ است و طول آن $\frac{1}{2}\textbf{v}$ است.
            بنابراین، از نظر هندسی، منفی کردن یک بردار را می‌توان به صورت "برگرداندن" جهت آن،
            و ضرب اسکالر را می توان به عنوان مقیاس بندی طول یک بردار در نظر گرفت.}

            \item {فرض کنید $\textbf{u}=(2,\frac{1}{2})$ و $\textbf{v}=(1,2)$. پس $\textbf{u}+\textbf{v}=(3,\frac{5}{2})$ .
            قسمت ب شکل \ref{fig:4.Session.1.1.6} نشان می‌دهد که جمع بردار از نظر هندسی به چه معناست:
            ما $\textbf{u}$ را به‌طور موازی انتقال میدهیم تا دم آن با سر $\textbf{v}$ منطبق شود.
            پس بردار مجموع ، برداری است که از دم $\textbf{v}$ شروع شده و با سر $\textbf{u}$ منتقل شده ختم می‌شود.
                (اگر $\textbf{u}$ را ثابت نگه داریم و $\textbf{v}$ را طوری انتقال دهیم که دم آن با سر $\textbf{u}$ منطبق شود، همین نتیجه را می گیریم.
                در این حالت، $\textbf{u}+\textbf{v}$ برداری خواهد بود که از دم $\textbf{u}$ شروع شده و با سر $\textbf{v}$ منتقل شده ختم میشود.)
                توجه داشته باشید که قوانین جمع بردار در زمانی که نیروها را با هم جمع می کنیم تا نیروی برآیند را به وجود بیاوریم، با انتظار ما مطابقت دارد:
                اگر دو نیرو (بردار) را در یک راستا اضافه کنیم، نیروی برآیند قوی تری (بردار طولانی تر) در آن جهت دریافت می کنیم.
                اگر دو نیرو (بردار) مخالف یکدیگر را بهم اضافه کنیم، نیروی برآیند ضعیف تری (بردار کوتاه تر) به دست می آید.
                شکل \ref{fig:4.Session.1.1.7} این ایده ها را نشان می دهد.}

            \item {فرض کنید $\textbf{v}=(2,1)$، $\textbf{v}$ و $\textbf{v}-\textbf{u}=(-1,\frac{3}{2})$. پس
            قسمت پ شکل \ref{fig:4.Session.1.1.6} نشان می دهد که تفریق برداری از نظر هندسی به چه معناست.
                $\textbf{v}-\textbf{u}$ به ما یک بردار با از سر $\textbf{u}$ تا سر $\textbf{v}$ می دهد.
                اگر در عوض u و v را به عنوان نقاط تفسیر کنیم، آنگاه $\textbf{v}-\textbf{u}$ برداری را با هدف از نقطه $\textbf{u}$ تا نقطه $\textbf{v}$ به ما می دهد.
                این تفسیر مهم است زیرا ما اغلب می خواهیم که بردار از یک نقطه به نقطه دیگر برود.
                همچنین طول $\textbf{v}-\textbf{u}$ فاصله $\textbf{u}$ تا $\textbf{v}$ است، وقتی که $\textbf{u}$ و $\textbf{v}$ را به عنوان نقطه در نظر بگیریم.}
        \end{enumerate}

        \begin{figure}[H]
            \centering
            \setlength{\belowcaptionskip}{-10pt}
            \includegraphics[width=\textwidth]{Images/4/4.Session.1.1.6}
            \caption{(آ) تفسیر هندسی ضرب اسکالر. (ب) تفسیر هندسی جمع بردار. ج) تفسیر هندسی تفریق بردار.}
            \label{fig:4.Session.1.1.6}
        \end{figure}

        \begin{figure}[H]
            \centering
            \setlength{\belowcaptionskip}{-10pt}
            \includegraphics[width=\textwidth]{Images/4/4.Session.1.1.7}
            \caption{نیروهای اعمال شده به یک توپ. نیروها با استفاده از جمع بردار ترکیب می شوند تا نیروی برآیند به دست آید.}
            \label{fig:4.Session.1.1.7}
        \end{figure}
    \end{spacing}
}

{
    \textbf{\BTitr بردارهای طول و واحد}

    \Large
    \begin{spacing}{1.5}
        از نظر هندسی اندازه ی یک بردار، طول پاره خطی جهت دار است.
        اندازه ی یک بردار را با خط های عمودی دوتایی نشان می دهیم
        (به عنوان مثال، $\norm{u}$ اندازه ی $\textbf{u}$ را نشان می دهد).
        حالا با توجه به بردار $\textbf{u}=(x,y,z)$، می‌خواهیم بزرگی آن را به صورت جبری محاسبه کنیم.
        اندازه ی یک بردار سه بعدی را می توان با دو بار اعمال قضیه فیثاغورث محاسبه کرد.
        شکل \ref{fig:4.Session.1.1.8} را ببینید. ابتدا، ما به مثلث در صفحه xz با اضلاع x، z و وتر a نگاه می کنیم.

        \begin{figure}[H]
            \centering
            \setlength{\belowcaptionskip}{-10pt}
            \includegraphics[width=0.5\textwidth]{Images/4/4.Session.1.1.8}
            \caption{طول سه بعدی یک بردار را می توان با دو بار اعمال قضیه فیثاغورث محاسبه کرد.}
            \label{fig:4.Session.1.1.8}
        \end{figure}

        از قضیه فیثاغورث، $a=\sqrt{\displaystyle x^2+z^2}$ داریم.
        حالا به مثلث با ضلع های \lr{a}، \lr{y} و وتر $\norm{u}$ نگاه کنید از قضیه فیثاغورث دوباره به فرمول اندازه ی زیر می رسیم:

        قضیه*****

        \begin{center}
            $\norm{\textbf{u}}=\sqrt{\displaystyle y^2+a^2}=\sqrt{\displaystyle y^2+(\sqrt{\displaystyle x^2+z^2})^2}=\sqrt{\displaystyle x^2+y^2+z^2}$
        \end{center}

        برای برخی از استفاده ها، ما به طول یک بردار اهمیتی نمی دهیم زیرا می خواهیم از بردار برای نمایش یک جهت خالص استفاده کنیم.
        می خواهیم طول چنین بردارهایی با فقط یک جهت دقیقاً 1 باشد.
        وقتی طول واحد برداری را می سازیم، می گوییم که بردار را نرمال کرده ایم.
        می توانیم یک بردار را با تقسیم هر یک از اجزای بر بزرگی آن ، نرمال کنیم:

        قضیه*****

        \begin{center}
            $\hat{\textbf{u}}=\frac{\displaystyle\textbf{u}}{\displaystyle\norm{\textbf{u}}}=\left(\frac{\displaystyle x}{\displaystyle\norm{\textbf{u}}},
            \frac{\displaystyle y}{\displaystyle\norm{\textbf{u}}}, \frac{\displaystyle z}{\displaystyle\norm{\textbf{u}}}\right)$
        \end{center}

        برای تأیید صحت این فرمول، می توانیم طول $\hat{\textbf{u}}$ را محاسبه کنیم

        \begin{center}
            $\norm{\hat{\textbf{u}}}=\sqrt{\left(\frac{\displaystyle x}{\displaystyle\norm{\textbf{u}}}\right)^2,
                \left(\frac{\displaystyle y}{\displaystyle\norm{\textbf{u}}}\right)^2,
                \left(\frac{\displaystyle z}{\displaystyle\norm{\textbf{u}}}\right)^2}=\frac{\displaystyle x^2+y^2+z^2}{\displaystyle\sqrt{\norm{\textbf{u}}^2}}
            =\frac{\displaystyle\norm{\textbf{u}}}{\displaystyle\norm{\textbf{u}}}=1$
        \end{center}


        پس \hat{\textbf{u}} در واقع یک بردار واحد است.

        مثال***

        بردار $\textbf{v}=(-1,3,4)$ را نرمال کنید. داریم $\norm{\textbf{v}}=\sqrt{\displaystyle (-1)^2+3^2+4^2}=\sqrt{26}$. پس:

        \begin{center}
            $\hat{\textbf{v}}=\frac{\displaystyle\textbf{v}}{\displaystyle\norm{\textbf{v}}}=\left(-\frac{\displaystyle 1}{\displaystyle\sqrt{26}},
            \frac{\displaystyle 3}{\displaystyle\sqrt{26}}, \frac{\displaystyle 4}{\displaystyle\sqrt{26}}\right)$
        \end{center}

        برای تأیید اینکه $\hat{\textbf{v}}$ واقعاً یک بردار واحد است، طول آن را محاسبه می کنیم:

        \begin{center}
            $\norm{\hat{\textbf{v}}}=\sqrt{\left(-\frac{\displaystyle 1}{\displaystyle\sqrt{26}}\right)^2,
                \left(\frac{\displaystyle 3}{\displaystyle\sqrt{26}}\right)^2, \left(\frac{\displaystyle 4}{\displaystyle\sqrt{26}}\right)^2}=
            \sqrt{\displaystyle\frac{\displaystyle 1}{\displaystyle 26}+\frac{\displaystyle 9}{\displaystyle 26}+\frac{\displaystyle 16}{\displaystyle 26}}=\sqrt{\displaystyle 1}=1$
        \end{center}

    \end{spacing}
}

{
    \textbf{\BTitr ضرب داخلی}

    \Large
    \begin{spacing}{1.5}
        ضرب نقطه ای (داخلی) شکلی از ضرب برداری است که منجر به یک مقدار اسکالر می شود.
        به همین دلیل، گاهی اوقات از آن به عنوان یک ضرب اسکالر یاد می شود.
        فرض کنید $\textbf{u}=(u_{x},u_{y},u_{z})$ و  $\textbf{v}=(v_{x},v_{y},v_{z})$،
        پس حاصل ضرب داخلی به صورت زیر تعریف می‌شود:

        \begin{center}
            $\textbf{u}\cdot\textbf{v}=u_{x}v_{x}+u_{y}v_{y}+u_{z}v_{z}$
        \end{center}

        به عبارت دیگر ، ضرب داخلی مجموع ضرب های اجزای مربوطه ست.

        همچنین تعریف ضرب داخلی معنای هندسی آشکاری ارائه نمی دهد.
        با استفاده از قانون کسینوس (به تمرین ***** مراجعه کنید)، می توانیم رابطه ای را پیدا کنیم که در آن $\theta$ زاویه بین بردارهای $\textbf{u}$ و $\textbf{v}$ است به طوری که $0\leq\theta\leq\pi$؛ شکل ****** را ببینید.

        \begin{center}
            $\textbf{u}\cdot\textbf{v}=\norm{\textbf{u}}\norm{\textbf{v}}\cos\theta$
        \end{center}

        بنابراین، معادله ***** می گوید که حاصلضرب نقطه ای بین دو بردار، کسینوس زاویه بین آنها است که با بزرگی بردارها ضرب شده است.
        به خصوص، اگر هر دو بردار $\textbf{u}$ و $\textbf{v}$ بردار واحد باشند،
        آنگاه $\textbf{u}\cdot\textbf{v}$ کسینوس زاویه بین آنهاست (یعنی $\textbf{u}\cdot\textbf{v}=\cos\theta$).
        معادله ***** برخی از ویژگی های هندسی مفید ضرب داخلی را در اختیار ما قرار می دهد:

        \begin{figure}[H]
            \centering
            \setlength{\belowcaptionskip}{-10pt}
            \includegraphics[width=0.5\textwidth]{Images/4/4.Session.1.1.9}
            \caption{در شکل سمت چپ، زاویه $\theta$ بین $\textbf{u}$ و $\textbf{v}$ یک زاویه تند است.
            در شکل سمت راست، زاویه $\theta$ بین $\textbf{u}$ و $\textbf{v}$ یک زاویه منفرجه (باز) است.
            وقتی به زاویه بین دو بردار اشاره می کنیم، همیشه منظورمان کوچکترین زاویه است،
            یعنی زاویه $\theta$ به طوری که $0\leq\theta\leq\pi$.}
            \label{fig:4.Session.1.1.9}
        \end{figure}

        \begin{enumerate}
            \item {اگر $\textbf{u}\cdot\textbf{v}=0$، در نتیجه $\textbf{u}\perp\textbf{v}$ (یعنی بردار ها بر هم عمودند).}
            \item {اگر $\textbf{u}\cdot\textbf{v}>0$، در نتیجه زاویه ی بین \theta و دو بردار کمتر از 90 درجه ست (یعنی بردار ها با هم زاویه ی تند میسازند).}
            \item {اگر $\textbf{u}\cdot\textbf{v}<0$، در نتیجه زاویه ی بین \theta و دو بردار بیشتر از 90 درجه ست (یعنی بردار ها با هم زاویه ی منفرجه میسازند).}
        \end{enumerate}

        مثال***
        فرض کنید $\textbf{u}=(1,2,3)$ و $\textbf{v}=(-4,0,-1)$ باشند. زوایه ی بین $\textbf{u}$ و $\textbf{v}$ را پیدا کنید.
        ابتدا موارد زیر را محاسبه میکنیم:

        \begin{center}
            $\textbf{u}\cdot\textbf{v}=(1,2,3)\cdot(-4,0,-1)=-4-3=-7$

            $\norm{\textbf{u}}=\sqrt{\displaystyle 1^2+2^2+3^2}=\sqrt{\displaystyle 14}$

            $\norm{\textbf{v}}=\sqrt{\displaystyle (-4)^2+0^2+(-1)^2}=\sqrt{\displaystyle 17}$
        \end{center}

        حالا ، قضیه ی ***** را برای حل \theta اعمال میکنیم:

        \begin{center}
            $\cos\theta=\frac{\displaystyle\textbf{u}\cdot\textbf{v}}{\displaystyle\norm{\textbf{v}}\norm{\textbf{u}}}=\frac{\displaystyle -7}{\displaystyle \sqrt{\displaystyle 14}\sqrt{\displaystyle 17}}$

            $\theta=\cos^-1\frac{\displaystyle -7}{\displaystyle \sqrt{\displaystyle 14}\sqrt{\displaystyle 17}}\thickapprox117^\circ$
        \end{center}


        مثال***

        شکل ***** را در نظر بگیرید.
        با توجه به $\textbf{v}$ و بردار واحد $\textbf{n}$، فرمولی برای $\textbf{p}$ بر حسب $\textbf{v}$ و $\textbf{n}$ با استفاده از ضرب داخلی پیدا کنید.
        ابتدا، از شکل مشاهده کنید که عدد اسکالر $\textbf{k}$ وجود دارد
        که $\textbf{p}=k\textbf{n}$؛ علاوه بر این، از آنجایی که ما $\norm{\textbf{u}}=1$ فرض کردیم،
        داریم $\norm{\textbf{p}}=\norm{k\textbf{n}}=\abs{k}\norm{\textbf{n}}=\abs{k}$.
        (توجه داشته باشید که $k$ ممکن است منفی باشد اگر و تنها اگر $\textbf{p}$ و $\textbf{n}$ در جهت مخالف قرار گیرند.)
        با استفاده از مثلثات، داریم که $k=\norm{\textbf{v}}\cos\theta$؛
        بنابراین، $\textbf{p}=k\textbf{n}=(\norm{\textbf{v}}\cos\theta)\textbf{n}$.
        با این حال، از آنجایی که \textbf{n} یک بردار واحد است، می توانیم این را به روش دیگری بگوییم:

        \begin{center}
            $\textbf{p}=(\norm{\textbf{v}}\cos\theta)\textbf{n}=(\norm{\textbf{v}}\cdot 1\cos\theta)\textbf{n}=(\norm{\textbf{v}}\norm{\textbf{n}}\cos\theta)\textbf{n}=(\textbf{v}\cdot\textbf{n})\textbf{n}$
        \end{center}

        به طور خاص، این $k=\textbf{v}\cdot\textbf{n}$ را نشان می دهد که تفسیر هندسی $\textbf{v}\cdot\textbf{n}$ را هنگامی که $\textbf{n}$ یک بردار واحد است، نشان می دهد. ما $\textbf{p}$ را تصویر متعامد/قائم (\lr{orthogonal projection}) $\textbf{v}$ روی $\textbf{n}$ می نامیم و معمولاً به شکل زیر نشان داده می شود:

        \begin{center}
            $\textbf{p}=proj_{n}(\textbf{v})$
        \end{center}

        اگر $\textbf{v}$ را به عنوان نیرو تعبیر کنیم، $\textbf{p}$ را می توان به عنوان بخشی از نیروی $\textbf{v}$ در نظر گرفت که در جهت $\textbf{n}$ عمل می کند.
        به همین ترتیب، بردار $\textbf{w}=perp_{n}(\textbf{v})=\textbf{v}-\textbf{p}$ بخشی از نیروی $\textbf{v}$ است که در جهت $\textbf{n}$ متعامد عمل می کند
        (به همین دلیل است که آن را با $perp_{n}(\textbf{v})$ برای عمود نشان می دهیم).
        مشاهده کنید که $\textbf{v}=\textbf{p}+\textbf{w}=proj_{n}(\textbf{v})+perp_{n}(\textbf{v})$ که می گوییم بردار $\textbf{v}$ را به مجموع دو بردار متعامد $\textbf{p}$ و $\textbf{w}$ تجزیه کرده ایم.

        اگر $\textbf{n}$ طول واحد نباشد، همیشه می‌توانیم ابتدا آن را نرمال کنیم تا طول آن را واحد کنیم.
        جایگزینی $\textbf{n}$ با بردار واحد $\frac{\displaystyle\textbf{n}}{\norm{\textbf{n}}}$ فرمول تصویر کلی تری را به ما می دهد:

        \begin{center}
            $\textbf{p}=proj_{n}(\textbf{v})=\left( \textbf{v}\cdot\frac{\displaystyle\textbf{n}}{\norm{\textbf{n}}} \right)\frac{\displaystyle\textbf{n}}{\norm{\textbf{n}}}=\frac{\displaystyle(\textbf{v}\cdot\textbf{n})}{\norm{\textbf{n}}^2}\textbf{n}$
        \end{center}

        \begin{figure}[H]
            \centering
            \setlength{\belowcaptionskip}{-10pt}
            \includegraphics[width=0.5\textwidth]{Images/4/4.Session.1.1.10}
            \caption{تصویر متعامد/قائم $\textbf{v}$ روی $\textbf{n}$.}
            \label{fig:4.Session.1.1.10}
        \end{figure}
    \end{spacing}
}

{
    \textbf{\BTitr متعامد سازی}

    \Large
    \begin{spacing}{1.5}
        مجموعه ای از بردارها ${\textbf{v}_{0},...,\textbf{v}_{n-1}}$ متعامد نامیده می شود که بردارها همگی متعامد (هر بردار در مجموعه نسبت به هر بردار دیگری در مجموعه متعامد باشد) و طول واحد باشند.
        گاهی اوقات ما مجموعه ای از بردارها را داریم که تقریباً متعارف هستند، اما نه کاملاً.
        یک کار متداول این است که مجموعه را متعامد سازی کنید تا یک مجموعه ی متعامد شود.
        در گرافیک کامپیوتری سه بعدی ممکن است با یک مجموعه متعارف شروع کنیم، اما به دلیل مسائل مربوط به دقت عددی، مجموعه به تدریج غیرمتعارف شود.
        ما عمدتاً با موارد دو بعدی و سه بعدی این مشکل (یعنی مجموعه هایی که به ترتیب شامل دو و سه بردار هستند) سروکار داریم.

        ابتدا مورد دو بعدی ساده تر را بررسی می کنیم.
        فرض کنید مجموعه بردارهای ${\textbf{v}_{0},\textbf{v}_{1}}$ را داریم که می‌خواهیم آن‌ها را به یک مجموعه متعامد ${\textbf{w}_{0},\textbf{w}_{1}}$ تبدیل کنیم،
        همانطور که در شکل ***** نشان داده شده است.
        ما با $\textbf{w}_{0}=\textbf{v}_{0}$ شروع کرده و $\textbf{v}_{1}$ را تغییر میدهیم تا آن را به $\textbf{w}_{0}$ متعامد کنیم.
        این کار با تفریق کردن بخشی از $\textbf{v}_{1}$ که در جهت $\textbf{w}_{0}$  عمل می کند، انجام می شود:

        \begin{center}
            $\textbf{w}_{1}=\textbf{v}_{1}-proj_{\textbf{w}_{0}}(\textbf{v}_{1})$
        \end{center}

        \begin{figure}[H]
            \centering
            \setlength{\belowcaptionskip}{-10pt}
            \includegraphics[width=0.5\textwidth]{Images/4/4.Session.1.1.11}
            \caption{متعامد سازی دو بعدی}
            \label{fig:4.Session.1.1.11}
        \end{figure}

        اکنون مجموعه ای متعامد از بردارها داریم ${\textbf{w}_{0},\textbf{w}_{1}}$.
        آخرین مرحله برای ساخت مجموعه متعارف نرمال کردن $\textbf{w}_{0}$ و $\textbf{w}_{1}$ برای ایجاد طول واحد است.

        حالت سه بعدی به همان شکل مورد دو بعدی اما با مراحل بیشتر است.
        فرض کنید مجموعه بردارهای ${\textbf{v}_{0},\textbf{v}_{1},\textbf{v}_{2}}$ را داریم که می‌خواهیم آنها را یک مجموعه متعامد ${\textbf{w}_{0},\textbf{w}_{1},\textbf{w}_{2}}$ کنیم، مطابق شکل *****.
        ما با $\textbf{w}_{0}=\textbf{v}_{0}$ شروع می کنیم و $\textbf{v}_{1}$ را تغییر می دهیم تا به $\textbf{w}_{0}$ متعامد شود.
        این کار با کم کردن بخشی از $\textbf{v}_{1}$ که در جهت $\textbf{w}_{0}$ عمل می کند انجام می شود:

        \begin{center}
            $\textbf{w}_{1}=\textbf{v}_{1}-proj_{\textbf{w}_{0}}(\textbf{v}_{1})$
        \end{center}

        \begin{figure}[H]
            \centering
            \setlength{\belowcaptionskip}{-10pt}
            \includegraphics[width=0.5\textwidth]{Images/4/4.Session.1.1.12}
            \caption{متعامد سازی سه بعدی}
            \label{fig:4.Session.1.1.12}
        \end{figure}

        در مرحله بعد، $\textbf{v}_{2}$ را تغییر می دهیم تا آن را متعامد به $\textbf{w}_{0}$ و $\textbf{w}_{1}$ کنیم.
        این کار با تفریق کردن بخشی از $\textbf{v}_{2}$ که در جهت $\textbf{w}_{0}$ عمل می کند و بخشی از $\textbf{v}_{2}$ که در جهت $\textbf{w}_{1}$ عمل می کند، انجام می شود:

        \begin{center}
            $\textbf{w}_{2}=\textbf{v}_{2}-proj_{\textbf{w}_{0}}(\textbf{v}_{2})-proj_{\textbf{w}_{1}}(\textbf{v}_{2})$
        \end{center}

        اکنون یک مجموعه متعامد از بردارها داریم ${\textbf{w}_{0},\textbf{w}_{1},\textbf{w}_{2}}$.
        آخرین مرحله برای ساخت مجموعه متعامد نرمال کردن $\textbf{w}_{0}$، $\textbf{w}_{1}$ و $\textbf{w}_{2}$ است تا طول آنها را واحد کنیم.

        برای حالت کلی $\textbf{n}$ بردار ${\textbf{v}_{0},...,\textbf{v}_{n-1}}$ که می‌خواهیم آنها را به یک مجموعه متعامد تبدیل کنیم { w0، ...، wn1}، ما روال زیر را داریم که معمولاً فرآیند متعامدسازی گرام اشمیت (\lr{Gram-Schmidt}) نامیده می شود:

        \begin{center}
            گام پایه: $\textbf{w}_{0}=\textbf{v}_{0}$ قرار میدهیم

            به ازای $1\leq i\leq n-1$ ، $\textbf{w}_{i}=\textbf{v}_{i}-\sum_{j=1}^{i-1}proj_{\textbf{w}_{j}}(\textbf{v}_{i})$ قرار میدهیم

            گام نرمال سازی: $\textbf{w}_{i}=\frac{\displaystyle\textbf{w}_{i}}{\displaystyle\norm{\textbf{w}_{i}}}$ قرار میدهیم
        \end{center}

        دوباره، ایده شهودی این است که وقتی یک بردار $\textbf{v}_{i}$ را از مجموعه ورودی انتخاب می کنیم تا به مجموعه متعارف اضافه کنیم،
        باید اجزای $\textbf{v}_{i}$ را که در جهت بردارهای دیگر عمل کرده و از قبل در مجموعه متعامد قرار دارند ($\textbf{w}_{0},\textbf{w}_{1},...,\textbf{w}_{i-1}$) را کم کنیم.
        این کار باعث اطمینان حاصل از این که بردار جدید اضافه شده با سایر بردارهای موجود در مجموعه متعامد، متعامد باشد، میشود.
    \end{spacing}
}

{
    \textbf{\BTitr ضرب خارجی}

    \Large
    \begin{spacing}{1.5}
        شکل دوم بردار ضرب ریاضی، ضرب خارجی است.
        بر خلاف ضرب داخلی که به یک اسکالر ختم می‌شود، ضرب خارجی به یک بردار دیگر ختم می‌شود.
        علاوه بر این، ضرب خارجی فقط برای بردارهای سه بعدی تعریف شده است (در واقع ضرب خارجی دو بعدی وجود ندارد).
        با ضرب خارجی دو بردار سه بعدی u و v، بردار دیگری به دست می‌آید،که w متعامد به u و v است.
        شکل ***** را ببینید.
        اگر $\textbf{u}=(u_{x},u_{y},u_{z})$ و  $\textbf{v}=(v_{x},v_{y},v_{z})$ باشد، آنگاه ضرب خارجی به این صورت محاسبه می‌شود:

        \begin{center}
            $\textbf{w}=\textbf{u}\times\textbf{v}=(u_{y}v_{z}-u_{z}v_{y}, u_{z}v_{x}-u_{x}v_{z}, u_{x}v_{y}-u_{y}v_{z})$
        \end{center}

        \begin{theo}{thm:pythagoras}
            \Large
            اگر در سیستم مختصات راستگرد کار می کنید، پس از قانون شست دست راست استفاده می کنید:
            اگر دست راست خود را طوری بگیرید که انگشتان در جهت بردار اول u باشد و سپس انگشتان خود را به سمت v با زاویه ی $0\leq\theata\leq \pi$ خم کنید، انگشت شست شما در جهت $\textbf{w}=\textbf{u}\times\textbf{v}$ قرار می گیرد.
        \end{theo}

        \begin{figure}[H]
            \centering
            \setlength{\belowcaptionskip}{-10pt}
            \includegraphics[width=0.5\textwidth]{Images/4/4.Session.1.1.13}
            \caption{ضرب خارجی دو بردار سه بعدی $\textbf{u}$ و $\textbf{v}$ بردار دیگر $\textbf{w}$ را به دست می آورد که متعامد با $\textbf{u}$ و $\textbf{v}$ است.
            اگر دست چپ خود را طوری بگیرید که انگشتان در جهت بردار اول u باشد و سپس انگشتان خود را به سمت v با زاویه ی $0\leq\theata\leq \pi$ خم کنید، انگشت شست شما در جهت $\textbf{w}=\textbf{u}\times\textbf{v}$ قرار می گیرد. به این قانون شست چپ می گویند.}
            {متعامد سازی سه بعدی}
            \label{Images/4/4.Session.1.1.13}
        \end{figure}

        مثال***

        فرض کنید $\textbf{u}=(1,2,3)$ و $\textbf{v}=(1,2,3)$ باشد. سپس $\textbf{w}=\textbf{u}\times\textbf{v}$ و $\textbf{z}=\textbf{v}\times\textbf{z}$ را محاسبه کنید. قضیه ی ***** را اعمال میکنیم:

        \begin{center}
            \begin{split}
                $\textbf{w}&=\textbf{u}\times\textbf{v}\\
                &=(2,1,3)\times(2,0,0)\\
                &=(1\cdot 0-3\cdot 0, 3\cdot 2-2\cdot 0, 2\cdot 0-1\cdot 2)\\
                &=(0,6,-2)$
            \end{split}
        \end{center}

        و

        \begin{center}
            \begin{split}
                $\textbf{z}&=\textbf{v}\times\textbf{u}\\
                &=(2,0,0)\times(2,1,3)\\
                &=(0\cdot 3-0\cdot 1, 0\cdot 2-2\cdot 3, 2\cdot 1-0\cdot 2)\\
                &=(0,6,-2)$
            \end{split}
        \end{center}

        این نتیجه یک چیز را روشن می کند، به طور کلی $\textbf{u}\times\textbf{v}\neq\textbf{v}\times\textbf{u}$. بنابراین، می گوییم که ضرب خارجی خاصیت جابجایی ندارد.
        در واقع می توان نشان داد که $\textbf{u}\times\textbf{v}=-\textbf{v}\times\textbf{u}$. شما می توانید بردار به دست آمده از ضرب خارجی را با قانون انگشت شست چپ تعیین کنید.
        اگر ابتدا انگشتان خود را در جهت بردار اول نشانه بگیرید و سپس انگشتان خود را به سمت بردار دوم خم کنید (همیشه مسیر را با کوچکترین زاویه انتخاب کنید)، همانطور که در شکل ***** نشان داده شده است، انگشت شست شما در جهت بردار حاصل قرار می گیرد.

        برای نشان دادن اینکه $\textbf{w}$ متعامد به $\textbf{u}$ و $\textbf{w}$ متعامد بر $\textbf{v}$ است،
        از ***** یادآوری می کنیم که اگر$\textbf{u}\times\textbf{v}=0$ باشد،
        آنگاه $\textbf{u}\perp\textbf{v}$ است (یعنی بردارها متعامد هستند). زیرا

        \begin{center}
            $\textbf{w}\times\textbf{u}=(0,6,-2)\cdot(2,1,3)=0\cdot 2+6\cdot 1+(-2)\cdot 3=0$

            و

            $\textbf{w}\times\textbf{v}=(0,6,-2)\cdot(2,0,0)=0\cdot 2+6\cdot 0+(-2)\cdot 0=0$
        \end{center}

        نتیجه می گیریم که $\textbf{w}$ متعامد به $\textbf{u}$ و $\textbf{w}$ متعامد بر $\textbf{v}$ است.
    \end{spacing}
}


{
    \textbf{\BTitr ضرب خارجی شبه دو بعدی}

    \Large
    \begin{spacing}{1.5}
        ضرب خارجی به ما امکان می دهد یک بردار متعامد به دو بردار سه بعدی داده شده را پیدا کنیم.
        در دوبعدی وضعیت مشابهی نداریم،
        اما با توجه به بردار دوبعدی $\textbf{u}=u_{x},u_{y}$ پیدا کردن بردار $\textbf{v}$ متعامد به $\textbf{u}$ می‌تواند مفید باشد.
        شکل ****** شکل هندسی ای را نشان می دهد که از آن $\textbf{v}=-u_{y},u_{x}$ پیدا میشود.
        اثبات آن ساده است:

        \begin{center}
            $\textbf{u}\cdot\textbf{v}=(u_{x},u_{y})\cdot(-u_{y},u_{x})=-u_{x}u_{y}+u_{y}u_{x}=0$
        \end{center}

        بنابراین $\textbf{u}\perp\textbf{v}$.
        مشاهده کنید که $\textbf{u}\cdot-\textbf{v}=u_{x}u_{y}+u_{y}(-u_{x})=0$، بنابراین $\textbf{u}\perp-\textbf{v}$ را نیز داریم.

        \begin{figure}[H]
            \centering
            \setlength{\belowcaptionskip}{-10pt}
            \includegraphics[width=0.5\textwidth]{Images/4/4.Session.1.1.14}
            \caption{ضرب خارجی شبه دوبعدی یک بردار $\textbf{u}$ به بردار متعامد $\textbf{v}$ ختم می شود.}
            \label{Images/4/4.Session.1.1.14}
        \end{figure}
    \end{spacing}
}


{
    \textbf{\BTitr متعامد سازی با ضرب خارجی}

    \Large
    \begin{spacing}{1.5}
        در ******، ما روشی برای متعامد کردن مجموعه ای از بردارها با استفاده از فرآیند گرام اشمیت را مشاهده کردیم.
        برای سه بعدی، استراتژی دیگری برای متعامد کردن مجموعه ای از بردارها $\textbf{v}_{0},\textbf{v}_{1},\textbf{v}_{2}$ وجود دارد که متعارف هستند،
        اما ممکن است به دلیل خطاهای انباشته شده در دقت عددی با استفاده از ضرب خارجی، غیرمتعارف شوند.
        برای شکل هندسی این فرآیند به شکل ****** مراجعه کنید:

        \begin{enumerate}
            \item {$\textbf{w}_{0}=\frac{\displaystyle\textbf{v}_{0}}{\displaystyle\norm{\textbf{v}_{0}}}$ قرار میدهیم}
            \item {$\textbf{w}_{2}=\frac{\displaystyle\textbf{w}_{0}\times\textbf{v}_{1}}{\displaystyle\norm{\textbf{w}_{0}\times\textbf{v}_{1}}}$ قرار میدهیم}
            \item {با قرار دادن $\textbf{w}_{1}=\textbf{w}_{2}\times\textbf{w}_{0}$ توسط تمرین ******، $\norm{\textbf{w}_{2}\times\textbf{w}_{0}}=1$ زیرا $\textbf{w}_{2}\perp\textbf{w}_{0}$ و $\norm{\textbf{w}_{2}}=\norm{\textbf{w}_{0}}=1$،
            بنابراین در این مرحله آخر نیازی به نرمال سازی نداریم.}
        \end{enumerate}

        در این مرحله، مجموعه بردارهای ${\textbf{w}_{0},\textbf{w}_{1},\textbf{w}_{2}}$ متعامد است.

        \begin{figure}[H]
            \centering
            \setlength{\belowcaptionskip}{-10pt}
            \includegraphics[width=0.5\textwidth]{Images/4/4.Session.1.1.15}
            \caption {متعامد سازی سه بعدی با ضرب خارجی}
            \label{Images/4/4.Session.1.1.15}
        \end{figure}

        \begin{theo}{thm:pythagoras}
            \Large
            در مثال بالا، ما با 00$\textbf{w}_{0}=\frac{\displaystyle\textbf{v}_{0}}{\displaystyle\norm{\textbf{v}_{0}}}$ شروع کردیم، به این معنی که هنگام رفتن از $\textbf{v}_{0}$ به $\textbf{w}_{0}$ جهت را تغییر ندادیم،
            ما فقط طول را تغییر دادیم.
            با این حال، جهت های $\textbf{w}_{1}$ و $\textbf{w}_{2}$ می تواند به ترتیب با $\textbf{v}_{1}$ و $\textbf{v}_{2}$ متفاوت باشد.
            بسته به کاربرد خاص، برداری که شما انتخاب میکنید تغییر جهت ندهد ممکن است مهم باشد.
            به عنوان مثال، در ادامه این کتاب، جهت دوربین را با سه بردار متعامد ${\textbf{v}_{0},\textbf{v}_{1},\textbf{v}_{2}}$ نشان می‌دهیم که بردار سوم $\textbf{v}_{2}$ جهتی را که دوربین به آن نگاه می‌کند، تعریف میکند.
            هنگامی که این بردارها را متعامد می کنیم، اغلب نمی خواهیم جهت مورد نظر خود را تغییر دهیم،
            بنابراین الگوریتم فوق را با $\textbf{v}_{2}$ شروع می کنیم و $\textbf{v}_{0}$ و $\textbf{v}_{1}$ را تغییر می دهیم تا بردارها متعامد شوند.
        \end{theo}
    \end{spacing}
}

{
    \textbf{\BTitr متعامد سازی با ضرب خارجی}

    \Large
    \begin{spacing}{1.5}
        تاکنون در مورد بردارها بحث کرده ایم که موقعیت ها را توصیف نمی کنند.
        با این حال، ما همچنین باید موقعیت هایی را در برنامه های سه بعدی خود مشخص کنیم،
        به عنوان مثال، موقعیت هندسی سه بعدی و موقعیت دوربین مجازی سه بعدی.
        نسبت به یک سیستم مختصات، می‌توانیم از یک بردار در موقعیت استاندارد (شکل *****) برای نمایش یک موقعیت سه بعدی در فضا استفاده کنیم.
        ما این بردار را بردار موقعیت می نامیم.
        در این حالت، محل نوک بردار مشخصه مورد نظر است، نه جهت یا بزرگی.
        ما از عبارات "بردار موقعیت" و "نقطه" به جای یکدیگر استفاده خواهیم کرد زیرا بردار موقعیت برای شناسایی یک نقطه کافی است.

        \begin{figure}[H]
            \centering
            \setlength{\belowcaptionskip}{-10pt}
            \includegraphics[width=0.5\textwidth]{Images/4/4.Session.1.1.16}
            \caption {بردار موقعیت، که از مبدا تا نقطه کشیده میشود، به طور کامل محل قرارگیری نقطه را نسبت به سیستم مختصات توصیف می کند.}
            \label{Images/4/4.Session.1.1.16}
        \end{figure}

        یکی از مزایای استفاده از بردارها برای نشان دادن نقاط، به ویژه در کد، این است که می‌توانیم عملیات های بردار را انجام دهیم که برای نقاط معنی ندارد.
        به عنوان مثال، از نظر هندسی، مجموع دو نقطه باید به چه معنا باشد؟
        از سوی دیگر، برخی از عملیات را می توان به نقاط گسترش داد.
        به عنوان مثال، ما فاصله ی دو نقطه $\textbf{q}-\textbf{p}$ را برداری از $\textbf{p}$ به $\textbf{q}$ تعریف می کنیم. همچنین، ما یک نقطه $\textbf{p}$ به اضافه بردار $\textbf{v}$ را نقطه ی $\textbf{q}$ تعریف میکنیم که با جابجایی p با بردار v به دست می آید.
        به راحتی، از آنجایی که ما از بردارها برای نشان دادن نقاط نسبت به یک سیستم مختصات استفاده می‌کنیم، نیازی به کار اضافی برای عملیات نقطه‌ای که قبلاً در مورد آن بحث شد، نیست، زیرا چارچوب جبر برداری از قبل آنها را آماده می‌کند. شکل ***** را ببینید.

        \begin{figure}[H]
            \centering
            \setlength{\belowcaptionskip}{-10pt}
            \includegraphics[width=0.5\textwidth]{Images/4/4.Session.1.1.17}
            \caption {(آ) فاصله ی دو نقطه $\textbf{q}-\textbf{p}$ برداری از $\textbf{p}$ به $\textbf{q}$ تعریف میشود. (ب) یک نقطه $\textbf{p}$ به اضافه بردار $\textbf{v}$ را نقطه ی $\textbf{q}$ تعریف میکنیم که با جابجایی p با بردار v به دست می آید.}
            \label{Images/4/4.Session.1.1.17}
        \end{figure}

        \begin{theo}{thm:pythagoras}
            \Large
            در واقع یک روش هندسی معنادار برای تعریف یک مجموع ویژه از نقاط وجود دارد که به آن ترکیب وابسته می گویند، که مانند میانگین وزنی نقاط است.
        \end{theo}
    \end{spacing}
}

{
    \textbf{\BTitr بردار های ریاضی \lr{DirectX}}

    \Large
    \begin{spacing}{1.5}
        در ویندوز 8 و بالاتر، \lr{DirectX Math} یک کتابخانه ریاضی سه بعدی برای برنامه \lr{Direct3D} است که بخشی از \lr{Windows SDK} است.
        این کتابخانه از مجموعه دستورالعمل \lr{SSE2} (\lr{Streaming SIMD Extensions 2}) استفاده می کند.
        با رجیسترهای \lr{SIMD} گسترده 128 بیتی (داده‌های چندگانه تک دستورالعمل)، دستورالعمل‌های \lr{SIMD} می‌توانند روی چهار عدد \lr{int} یا عدد \lr{float} 32 بیتی با یک دستورالعمل کار کنند.
        این برای محاسبات برداری بسیار مفید است. برای مثال، اگر به جمع برداری نگاه کنید:

        \begin{center}
            $\textbf{u}+\textbf{v}=(u_{x}+v_{x},u_{y}+v_{y},u_{z}+v_{z})$
        \end{center}

        می بینیم که ما فقط اجزای مربوطه را اضافه می کنیم.
        با استفاده از \lr{SIMD} می‌توانیم به جای چهار دستور اسکالر، جمع بردار چهار بعدی را با یک دستور \lr{SIMD} انجام دهیم.
        اگر ما فقط به سه مختصات برای کار سه بعدی نیاز داشته باشیم، همچنان می توانیم از \lr{SIMD} استفاده کنیم، اما مختصات چهارم را نادیده می گیریم.
        به همین ترتیب، برای دو بعدی، مختصات سوم و چهارم را نادیده می گیریم.

        پوشش ما از کتابخانه ریاضی \lr{DirectX} جامع نیست و ما فقط بخش های کلیدی مورد نیاز برای این کتاب را پوشش می دهیم.
        برای تمام جزئیات، ما مستندات آنلاین [\lr{DirectXMath}] را توصیه می کنیم.
        برای خوانندگانی که مایلند بدانند چگونه یک کتابخانه برداری \lr{SIMD} ممکن است به طور بهینه توسعه یابد، و شاید درک کنند که چرا کتابخانه ریاضی \lr{DirectX} برخی از تصمیمات طراحی را اتخاذ کرده است،
        مقاله \lr{Designing Fast Cross-Platform SIMD Vector Libraries [by Oliveira2010]} را توصیه می کنیم.

        برای استفاده از کتابخانه ریاضی \lr{DirectX}، باید \texttt{#include <DirectXMath.h>} و برای برخی از انواع داده های اضافی \texttt{#include <DirectXPackedVector.h>} را وارد کنید.
        هیچ فایل کتابخانه ای اضافی وجود ندارد، زیرا تمام کدها به صورت درون خطی در فایل هدر پیاده سازی می شوند.
        کد \lr{DirectXMath.h} در فضای نام \texttt{DirectX} و کد \lr{DirectXPackedVector.h} در فضای نام \texttt{DirectX::PackedVector} وجود دارد.
        علاوه بر این، برای پلتفرم \lr{x86} باید \lr{SSE2} را فعال کنید:

        (\textbf{Project Properties > Configuration Properties > C/C++ > Code Generation > Enable Enhanced Instruction Set})

        و برای همه پلتفرم‌ها باید مدل نقطه شناور سریع \lr{/fp:fast} را فعال کنید:

        (\textbf{Project Properties > Configuration Properties > C/C++ > Code Generation > Floating Point Model})

        شما نیازی به فعال کردن \lr{SSE2} برای پلتفرم \lr{x64} ندارید زیرا تمام پردازنده‌های \lr{x64} از \lr{SSE2} (\url{http://en.wikipedia.org/wiki/SSE2}) پشتیبانی می‌کنند.
    \end{spacing}
}