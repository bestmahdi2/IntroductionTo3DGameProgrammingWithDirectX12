%-----------------------------------------------------------------------------------------------------------%
\newpage

\setcounter{chapter}{3}
\setcounter{example}{0}
\setcounter{eqtn}{0}
\setcounter{section}{0}


\chapter{\lr{\textbf{تبدیل ها}}}
\textbf{\vspace{-140pt}}
\begin{figure}[H]
    \centering
    \setlength{\belowcaptionskip}{-10pt}
    \includegraphics[width=0.8\textwidth]{Images/4/3/4.Session.1.3.0}
    \label{fig:4.Session.1.3.0}
\end{figure}
\textbf{\vspace{20pt}}
{
    \Large
    \begin{spacing}{1.5}
        ما اشیاء موجود در جهان های سه بعدی خود را به صورت هندسی توصیف می کنیم.
        یعنی به صورت مجموعه ای از مثلث هایی که با سطوح بیرونی اجسام تقریب دارند.
        اگر اشیاء ما بی حرکت بمانند، دنیای غیر جالبی خواهیم داشت.
        بنابراین ما به روش هایی برای تبدیل هندسی علاقه مند هستیم.
        نمونه‌هایی از تبدیل‌های هندسی عبارتند از: انتقال، دوران و مقیاس‌بندی.
        در این فصل، معادلات ماتریسی را توسعه می‌دهیم که می‌توان از آنها برای تبدیل نقاط و بردارها در فضای سه‌بعدی استفاده کرد.
        \\

        \textbf{\LARGE \hspace{-40pt}اهداف:}
        \begin{enumerate}[label=\textbf{\arabic*}.]
            \item {درک چگونگی نشان دادن تبدیل های خطی و وابسته با ماتریس ها.}
            \item {یادگیری تبدیل مختصات برای مقیاس بندی، دوران و انتقال هندسی.}
            \item {کشف چگونگی ترکیب کردن چندین ماتریس تبدیل به یک ماتریس تبدیل خالص از طریق ضرب ماتریس-ماتریس.}
            \item {فهمیدن چگونگی تبدیل سیستم مختصات به سیستم مختصات دیگر و چگونگی نشان دادن این تغییر تبدیل مختصات با یک ماتریس.}
            \item {آشنایی با زیر مجموعه توابع ارائه شده توسط کتابخانه ریاضی \lr{DirectX} که برای ساخت ماتریس های تبدیل استفاده می شوند.}
        \end{enumerate}
    \end{spacing}
}
%-----------------------------------------------------------------------------------------------------------%
\newpage

\setcounter{figure}{0}
\renewcommand{\thefigure}{\arabic{figure}.\arabic{chapter}}


\section{\textbf{تبدیلات خطی}}
\label{sec:3.1}
\subsection{\textbf{تعریف}}
{
    \Large
    \begin{spacing}{1.5}
        تابع ریاضی $\tau(\textbf{v})=\tau(x,y,z)=(x\prime,y\prime,z\prime)$ را در نظر بگیرید.
        این تابع یک بردار سه بعدی را دریافت کرده و یک بردار سه بعدی را خروجی می دهد.
        می گوییم که $\tau$ یک تبدیل خطی است اگر و تنها اگر که ویژگی های زیر برقرار باشد:

        \begin{eqtn}{eqtn:3.1}
            \centering
            $\tau(\textbf{u}+\textbf{v})=\tau(\textbf{u})\tau(\textbf{v})$\\
            $\tau(k\textbf{u})=k\tau(\textbf{u})$
        \end{eqtn}

        که در آن $\textbf{u}=(u_x,u_y,u_z)$ و $\textbf{v}=(v_x,v_y,v_z)$ بردار سه بعدی هستند و $k$ یک اسکالر است.

        \begin{point}{pnt:3.1}
            یک تبدیل خطی می تواند شامل مقادیر ورودی و خروجی غیر از بردارهای سه بعدی باشد،
            اما ما در کتاب گرافیک سه بعدی نیازی به چنین کلیتی نداریم.
        \end{point}

        \begin{example}{exp:3.1}
            \Large
            تابع $\tau(x,y,z)=(x^2,y^2,z^2)$ را تعریف کنید.
            به عنوان مثال، $\tau(1,2,3)=(1,3,9)$. این تابع خطی نیست زیرا برای $k=2$ و $\textbf{u}=(1,2,3)$، داریم:

            \begin{center}
                $\tau(k\textbf{u})=\tau(2,4,6)=(4,16,36)$
            \end{center}

            اما

            \begin{center}
                $k\tau(\textbf{u})=2(1,4,9)=(2,8,18)$
            \end{center}

            بنابراین خاصیت 2 معادله \ref{eqtn:3.1} برآورده نمی شود.
            اگر $\tau$ خطی باشد، نتیجه می شود که:

            \begin{eqtn}{eqtn:3.2}
                \centering
                \begin{equation*}
                    \centering
                    \begin{split}
                        \tau(a\textbf{u}+b\textbf{v}+c\textbf{w})&=\tau(a\textbf{u}+(b\textbf{v}+c\textbf{w})) \\
                        &=a\tau(\textbf{u})+\tau(b\textbf{v}+c\textbf{w})\\
                        &=a\tau(\textbf{u})+b\tau(\textbf{v})+c\tau(\textbf{w})
                    \end{split}
                \end{equation*}
            \end{eqtn}

            در بخش بعدی از این نتیجه استفاده خواهیم کرد.
        \end{example}
    \end{spacing}
}

\subsection{\textbf{نمایش ماتریسی}}
{
    \Large
    \begin{spacing}{1.5}
        فرض کنید $\textbf{u}=(x,y,z)$.
        همیشه می توانیم این را به صورت زیر بنویسیم:

        \begin{center}
            $\textbf{u}=(x,y,z)=x\textbf{i}+y\textbf{j}+z\textbf{k}=x(1,0,0)+y(0,1,0)+z(0,0,1)$
        \end{center}

        بردارهای $\textbf{i}=(1,0,0)$، $\textbf{j}=(0,1,0)$، و $\textbf{k}=(0,0,1)$، که بردارهای واحدی هستند
        که به ترتیب در امتداد محورهای مختصات کاری هدف قرار می گیرند، بردارهای پایه استاندارد برای $\mathbb{R}^3$ نامیده می شوند.
        ($\mathbb{R}^3$ مجموعه تمام بردارهای مختصات سه بعدی (x، y، z) را نشان می دهد).
        حال اجازه دهید $\tau$ یک تبدیل خطی باشد.
        با خطی بودن (یعنی معادله \ref{eqtn:3.2})، داریم:

        \begin{figure}[H]
            \centering
            \setlength{\belowcaptionskip}{-10pt}
            \includegraphics[width=0.5\textwidth]{Images/4/3/4.Session.1.3.1}
            \caption {سرباز سمت چپ شی اصلی است.
            سرباز میانی همان سرباز اصلی است که 2 واحد در محور y مقیاس دارد و آن را بلندتر می کند.
            سرباز سمت راست همان سرباز اصلی است که 2 واحد در محور x مقیاس دارد و آن را چاق تر می کند.}
            \label{fig:4.Session.1.3.1}
        \end{figure}

        \begin{eqtn}{eqtn:3.3}
            \centering
            $\tau(\textbf{u})=\tau(x\textbf{i}+y\textbf{j}+z\textbf{k})=x\tau(\textbf{i})+y\tau(\textbf{j})+z\tau(\textbf{k})$
        \end{eqtn}

        \begin{eqtn}{eqtn:3.4}
            \centering

            \begin{equation*}
                \centering
                \begin{split}
                    \tau(\textbf{u})&=x\tau(\textbf{i})+y\tau(\textbf{j})+z\tau(\textbf{k}) \\
                    &=\textbf{uA}=[x,y,z]\begin{bmatrix}
                                             \leftarrow & \tau(\textbf{i}) & \rightarrow \\
                                             \leftarrow & \tau(\textbf{j}) & \rightarrow \\
                                             \leftarrow & \tau(\textbf{k}) & \rightarrow
                    \end{bmatrix}=[x,y,z]\begin{bmatrix}
                                             A_{11} & A_{12} & A_{13} \\
                                             A_{21} & A_{22} & A_{23} \\
                                             A_{31} & A_{32} & A_{33}
                    \end{bmatrix}
                \end{split}
            \end{equation*}
        \end{eqtn}

        که در آن $\tau(\textbf{i})=(A_{11},A_{12},A_{13})$، $\tau(\textbf{j})=(A_{21},A_{22},A_{23})$، و $\tau(\textbf{k})=(A_{31},A_{32},A_{33})$.
        ماتریس $\textbf{A}$ را نمایش ماتریسی تبدیل خطی $\tau$ می نامیم.
    \end{spacing}
}

\subsection{\textbf{مقیاس بندی}}
{
    \Large
    \begin{spacing}{1.5}
        مقیاس بندی به تغییر اندازه یک شی اشاره دارد که در شکل \ref{fig:4.Session.1.3.1} نشان داده شده است.
        ما تحول مقیاس گذاری را توسط:

        \begin{center}
            $S(x,y,z)=(s_{x}x,s_{y}y,s_{z}z)$
        \end{center}

        این بردار را با واحدهای $s_x$ در محور $x$، واحدهای $s_y$ در محور $y$، و واحدهای $s_z$ در محور $z$،
        نسبت به مبدأ سیستم مختصات کار مقیاس بندی می‌کند.
        اکنون نشان می دهیم که $S$ در واقع یک تبدیل خطی است. ما داریم که:

        \begin{equation*}
            \centering
            \begin{split}
                S(\textbf{u}+\textbf{v})&=\left( s_{x}(u_{x}+v_{x}), s_{y}(u_{y}+v_{y}), s_{z}(u_{z}+v_{z}) \right) \\
                &=(s_{x}u_{x}+s_{x}v_{x}, s_{y}u_{y}+s_{y}v_{y}, s_{z}u_{z}+s_{z}v_{z}) \\
                &=(s_{x}u_{x}, s_{y}u_{y}, s_{z}u_{z})+(s_{x}v_{x}, s_{y}v_{y}, s_{z}v_{z}) \\
                &=S(\textbf{u})+S(\textbf{v}) \\
                S(k\textbf{u})&=(s_{x}ku_{x}, s_{y}ku_{y}, s_{z}ku_{z}) \\
                &=K(s_{x}u_{x}, s_{y}u_{y}, s_{z}u_{z}) \\
                &=KS(\textbf{u}) \\
            \end{split}
        \end{equation*}

        بنابراین هر دو ویژگی معادله \ref{eqtn:3.1} برآورده شدند، بنابراین $S$ خطی است،
        و بنابراین یک نمایش ماتریسی وجود دارد. برای یافتن نمایش ماتریس، ما فقط $S$ را به هر یک از بردارهای پایه استاندارد، مانند رابطه \ref{eqtn:3.3} اعمال می کنیم،
        و سپس بردارهای حاصل را در ردیف های یک ماتریس قرار می دهیم (مانند معادله \ref{eqtn:3.4}):

        \begin{center}
            $S(\textbf{i})=(s_x\cdot 1,s_y\cdot 0,s_z\cdot 0)=(s_x,0,0) \\
            S(\textbf{j})=(s_x\cdot 0,s_y\cdot 1,s_z\cdot 0)=(0,s_y,0) \\
            S(\textbf{k})=(s_x\cdot 0,s_y\cdot 0,s_z\cdot 1)=(0,0,s_z)$
        \end{center}

        بنابراین نمایش ماتریسی $S$ به صورت زیر است:

        \begin{center}
            $\textbf{S}=\begin{bmatrix}
                            s_x & 0   & 0   \\
                            0   & s_y & 0   \\
                            0   & 0   & s_z
            \end{bmatrix}$
        \end{center}

        ما این ماتریس را ماتریس مقیاس بندی می نامیم.
        معکوس ماتریس مقیاس بندی به صورت زیر بدست می آید:

        \begin{center}
            $\textbf{S}^{-1}=\begin{bmatrix}
                                 1/s_x & 0     & 0     \\
                                 0     & 1/s_y & 0     \\
                                 0     & 0     & 1/s_z
            \end{bmatrix}$
        \end{center}

        \begin{example}{exp:3.2}
            \Large
            فرض کنید مربعی داریم که با یک نقطه حداقل $(-4,-4, 0)$ و یک نقطه حداکثر $(4,4,0)$ تعریف شده است.
            اکنون فرض کنید که می خواهیم مربع را $0.5$ واحد در محور $x$، $2.0$ واحد در محور $y$، و محور $z$ را بدون تغییر رها کنیم.
            ماتریس مقیاس بندی مربوطه به صورت زیر است:

            \begin{center}
                $\textbf{S}=\begin{bmatrix}
                                0.5 & 0 & 0 \\
                                0   & 2 & 0 \\
                                0   & 0 & 1
                \end{bmatrix}$
            \end{center}

            \begin{figure}[H]
                \centering
                \setlength{\belowcaptionskip}{-10pt}
                \includegraphics[width=0.8\textwidth]{Images/4/3/4.Session.1.3.2}
                \caption {مقیاس بندی به نیم واحد در محور x و دو واحد در محور y. توجه داشته باشید که هنگام نگاه کردن به محور z منفی، هندسه 2 بعدی است زیرا z=0 است.}
                \label{fig:4.Session.1.3.2}
            \end{figure}

            اکنون برای مقیاس (تبدیل) مربع، هر دو نقطه حداقل و حداکثر را در این ماتریس ضرب می کنیم:

            \begin{center}
                $[-4,-4,0]\begin{bmatrix}
                              0.5 & 0 & 0 \\
                              0   & 2 & 0 \\
                              0   & 0 & 1
                \end{bmatrix}=[-2,-8,0]$\hspace{5 mm}$[4,4,0]=\begin{bmatrix}
                                                                  0.5 & 0 & 0 \\
                                                                  0   & 2 & 0 \\
                                                                  0   & 0 & 1
                \end{bmatrix}=[2,8,0]$
            \end{center}

            نتیجه در شکل \ref{fig:4.Session.1.3.2} نشان داده شده است.
        \end{example}
    \end{spacing}
}

\subsection{\textbf{دوران}}
{
    \Large
    \begin{spacing}{1.5}
        در این بخش، دوران بردار $\textbf{v}$ حول محور $\textbf{n}$ با زاویه $\theta$ را شرح می دهیم.
        شکل \ref{fig:4.Session.1.3.3} را ببینید.
        توجه داشته باشید که هنگام نگاه کردن به محور $\textbf{n}$ زاویه را در جهت عقربه های ساعت اندازه می گیریم.
        علاوه بر این، فرض می کنیم $\norm{\textbf{n}}=1$.

        \begin{figure}[H]
            \centering
            \setlength{\belowcaptionskip}{-10pt}
            \includegraphics[width=0.8\textwidth]{Images/4/3/4.Session.1.3.3}
            \caption {هندسه دوران حول بردار $\textbf{n}$.}
            \label{fig:4.Session.1.3.3}
        \end{figure}

        ابتدا $\textbf{v}$ را به دو قسمت تجزیه کنید:
        یک قسمت موازی با $\textbf{n}$ و قسمت دیگر متعامد با $\textbf{n}$.
        قسمت موازی فقط projn(v) است (مثال \ref{exp:1.5} را به خاطر بیاورید).
        قسمت متعامد با $\textbf{v}_{\perp}=perp_{n}(\textbf{v})=\textbf{v}-proj_{n}(\textbf{v})$ داده می شود.
        (همچنین از مثال \ref{exp:1.5} به یاد بیاورید که از آنجایی که $\textbf{n}$ یک بردار واحد است، ما $proj_{n}(\textbf{v})= (\textbf{n}\cdot\textbf{v})\textbf{n}$ را داریم.)
        مشاهدات کلیدی این است که بخشی $proj_{n}(\textbf{v})$ که موازی با $\textbf{n}$ است ، تحت دوران ثابت است.
        بنابراین ما فقط باید بفهمیم که چگونه قسمت متعامد را دوران دهیم. یعنی بردار چرخشی $R_{n}(\textbf{v})=proj_{n}(\textbf{v})+R_{n}(\textbf{v}_{\perp})$، در شکل \ref{fig:4.Session.1.3.3}.
        برای یافتن $R_{n}(\textbf{v}_{\perp})$، یک سیستم مختصات دوبعدی را در صفحه دوران تنظیم کردیم.
        ما از $\textbf{v}_{\perp}$ به عنوان یک بردار مرجع استفاده خواهیم کرد.
        برای به دست آوردن یک بردار مرجع دوم متعامد به $\textbf{v}_{\perp}$ و $\textbf{n}$،
        حاصل ضرب متقاطع $\textbf{n}\times\textbf{v}$ (قانون شست دست چپ) را می گیریم. از مثلثات شکل \ref{fig:4.Session.1.3.3} و تمرین \ref{question14} از فصل 1، می بینیم که

        \begin{center}
            $\norm{\textbf{n}\times\textbf{v}}=\norm{\textbf{n}}\norm{\textbf{v}}\sin\alpha=\norm{\textbf{v}}\sin\alpha=\norm{\textbf{v}_{\perp}}$
        \end{center}

        که در آن $\alpha$ زاویه بین $\textbf{n}$ و $\textbf{n}$ است.
        بنابراین هر دو بردار مرجع دارای طول یکسانی هستند و روی دایره دوران قرار می گیرند.
        اکنون که این دو بردار مرجع را تنظیم کردیم، از مثلثات می بینیم که:

        \begin{center}
            $R_{n}(\textbf{v}_{\perp})=\cos\theta\textbf{v}_{\perp}+sin\theta(\textbf{n}\times\textbf{v})$
        \end{center}

        این فرمول دوران زیر را به ما می دهد:

        \begin{eqtn}{eqtn:3.5}
            \centering
            \begin{equation*}
                \centering
                \begin{split}
                    R_{n}(\textbf{v})&=proj_{n}(\textbf{v})+R_{n}(\textbf{v}_{\perp})\\
                    &=(\textbf{n}\cdot\textbf{v})\textbf{n}+\cos\theta\textbf{v}_{\perp}+\sin\theta(\textbf{n}\times\textbf{v})\\
                    &=(\textbf{n}\cdot\textbf{v})\textbf{n}+\cos\theta(\textbf{v}-(\textbf{n}\cdot\textbf{v})\textbf{n})+\sin\theta(\textbf{n}\times\textbf{v})\\
                    &=\cos\theta\textbf{v}+(1-\cos\theta)(\textbf{n}\cdot\textbf{v})\textbf{n}+\sin\theta(\textbf{n}\times\textbf{v})
                \end{split}
            \end{equation*}
        \end{eqtn}

        ما آن را به عنوان تمرینی رها می کنیم تا نشان دهیم که این یک تبدیل خطی است.
        برای یافتن نمایش ماتریس، ما فقط $R_{n}$ را به هر یک از بردارهای پایه استاندارد مانند معادله \ref{eqtn:3.3} اعمال می کنیم و سپس بردارهای حاصل را در ردیف های یک ماتریس قرار می دهیم (مانند معادله \ref{eqtn:3.4}).
        نتیجه نهایی این است:

        \begin{center}
            $\textbf{R}_{n}=\begin{bmatrix}
                                c+(1-c)x^{2} & (1-c)xy+sz   & (1-c)xz-sy \\
                                (1-c)xy-sz   & c+(1-c)y^{2} & (1-c)yz+sx \\
                                (1-c)xz+sy   & (1-c)yz-sx   & c+(1-c)z^{2}
            \end{bmatrix}$
        \end{center}

        جایی که $c=\cos\theta$ و $s=\sin\theta$ را فرض میکنیم.
        ماتریس های دوران خاصیت جالبی دارند. هر بردار ردیف واحد طول (تأیید) و بردارهای ردیف متعامد (تأیید) هستند.
        بنابراین بردارهای ردیف متعامد هستند (یعنی متعامد متعامد و طول واحد).
        به ماتریسی که ردیف‌های آن متعامد هستند، ماتریس متعامد گفته می‌شود.
        یک ماتریس متعامد دارای خاصیت جذابی است که معکوس آن در واقع برابر با جابجایی آن است. بنابراین، معکوس $R_{n}$ برابر است با:

        \begin{center}
            $\textbf{R}^{-1}_{n}=\textbf{R}^{T}_{n}=\begin{bmatrix}
                                                        c+(1-c)x^{2} & (1-c)xy-sz   & (1-c)xz+sy \\
                                                        (1-c)xy+sz   & c+(1-c)y^{2} & (1-c)yz-sx \\
                                                        (1-c)xz-sy   & (1-c)yz+sx   & c+(1-c)z^{2}
            \end{bmatrix}$
        \end{center}

        به طور کلی، ماتریس های متعامد برای کار با آنها مطلوب هستند، زیرا معکوس آنها برای محاسبه آسان و کارآمد است.
        به ویژه، اگر محورهای $x$، $y$ و $z$ را برای چرخش انتخاب کنیم (یعنی به ترتیب $\textbf{n}=(1,0,0)$، $\textbf{n}=(0,1,0)$ و $\textbf{n}=(0,0,1)$)،
        سپس ماتریس های دوران زیر را دریافت می کنیم که به ترتیب حول محور $x$، $y$ و $z$ دوران میکنند:

        \begin{center}
            $\textbf{R}_{x}=\begin{bmatrix}
                                1 & 0           & 0          & 0 \\
                                0 & \cos\theta  & \sin\theta & 0 \\
                                0 & -\sin\theta & \cos\theta & 0 \\
                                0 & 0           & 0          & 1
            \end{bmatrix}, \textbf{R}_{y}=\begin{bmatrix}
                                              \cos\theta & 0 & -\sin\theta & 0 \\
                                              0          & 1 & 0           & 0 \\
                                              \sin\theta & 0 & \cos\theta  & 0 \\
                                              0          & 0 & 0           & 1
            \end{bmatrix}, \textbf{R}_{z}=\begin{bmatrix}
                                              \cos\theta  & \sin\theta & 0 & 0 \\
                                              -\sin\theta & \cos\theta & 0 & 0 \\
                                              0           & 0          & 1 & 0 \\
                                              0           & 0          & 0 & 1
            \end{bmatrix}$
        \end{center}

        \begin{example}{exp:3.3}
            \Large
            فرض کنید مربعی داریم که با یک نقطه حداقل $(-1, 0, -1)$ و یک نقطه حداکثر $(1, 0, 1)$ تعریف شده است.
            حال فرض کنید که می‌خواهیم مربع $-30^\circ$ را در جهت عقربه‌های ساعت حول محور $y$ دوران دهیم (یعنی $30^\circ$ خلاف جهت عقربه‌های ساعت).
            در این مورد، $\textbf{n}=(0,1,0)$، که $\textbf{R}_{n}$ را به طور قابل توجهی ساده می کند.
            ماتریس دوران محور $y$  مربوطه به صورت زیر است:

            \begin{center}
                $\textbf{R}_{y}=\begin{bmatrix}
                                    \cos\theta & 0 & -\sin\theta \\
                                    0          & 1 & 0           \\
                                    \sin\theta & 0 & \cos\theta
                \end{bmatrix}=\begin{bmatrix}
                                  \cos(-30^\circ) & 0 & -\sin(-30^\circ) \\
                                  0               & 1 & 0                \\
                                  \sin(-30^\circ) & 0 & \cos(-30^\circ)
                \end{bmatrix}=\begin{bmatrix}
                                  \frac{\displaystyle \sqrt{\displaystyle 3}}{\displaystyle 2} & 0 & \frac{\displaystyle 1}{\displaystyle 2}                      \\
                                  0                                                            & 1 & 0                                                            \\
                                  \frac{\displaystyle 1}{\displaystyle 2}                      & 0 & \frac{\displaystyle \sqrt{\displaystyle 3}}{\displaystyle 2}
                \end{bmatrix}$
            \end{center}

            اکنون برای دوران (تبدیل) مربع، هر دو نقطه حداقل و حداکثر را در این ماتریس ضرب می کنیم:

            \begin{center}
                $[-1,0,-1]\begin{bmatrix}
                              \frac{\displaystyle \sqrt{\displaystyle 3}}{\displaystyle 2} & 0 & \frac{\displaystyle 1}{\displaystyle 2}                      \\
                              0                                                            & 1 & 0                                                            \\
                              \frac{\displaystyle 1}{\displaystyle 2}                      & 0 & \frac{\displaystyle \sqrt{\displaystyle 3}}{\displaystyle 2}
                \end{bmatrix}\approx[-0.36,0,-1.36]\hspace{5 mm}[1,0,1]\begin{bmatrix}
                                                                           \frac{\displaystyle \sqrt{\displaystyle 3}}{\displaystyle 2} & 0 & \frac{\displaystyle 1}{\displaystyle 2}                      \\
                                                                           0                                                            & 1 & 0                                                            \\
                                                                           \frac{\displaystyle 1}{\displaystyle 2}                      & 0 & \frac{\displaystyle \sqrt{\displaystyle 3}}{\displaystyle 2}
                \end{bmatrix}\approx[0.36,0,1.36]$
            \end{center}

            نتیجه در شکل \ref{fig:4.Session.1.3.4} نشان داده شده است.

            \begin{figure}[H]
                \centering
                \setlength{\belowcaptionskip}{-10pt}
                \includegraphics[width=0.8\textwidth]{Images/4/3/4.Session.1.3.4}
                \caption {دوران $-30^\circ$ در جهت عقربه های ساعت حول محور $y$.
                توجه داشته باشید که هنگام نگاه کردن به محور مثبت $y$، هندسه اساساً 2 بعدی است زیرا $y=0$ است.}
                \label{fig:4.Session.1.3.4}
            \end{figure}
        \end{example}
    \end{spacing}
}


\section{\textbf{تبدیلات آفین}}
\label{sec:3.2}
\subsection{\textbf{مختصات همگن}}
{
    \Large
    در بخش بعدی خواهیم دید که تبدیل آفین یک تبدیل خطی همراه با انتقال است.
    با این حال، انتقال برای بردارها معنی ندارد زیرا یک بردار فقط جهت و اندازه را، مستقل از مکان، توصیف می کند.
    به عبارت دیگر، بردارها باید تحت انتقال بدون تغییر باشند.
    انتقال ها فقط باید برای نقاط (به عنوان مثال، بردارهای موقعیت) اعمال شوند.
مختصات همگن مکانیسم نمادگذاری مناسبی را فراهم می کند که ما را قادر می سازد نقاط و بردارها را به طور یکنواخت مدیریت کنیم.
با مختصات همگن، ما به تاپل $4$ تایی منتقل میشویم و آنچه را که در چهارمین مختصات $w$ قرار می دهیم بستگی به این دارد که یک نقطه یا بردار را توصیف کنیم.
به طور خاص می نویسیم:
    \begin{spacing}{1.5}
        \lr{
            \begin{enumerate}[label=\textbf{\arabic*}.]
                \item {(x,y,z,0) \rl{برای بردار ها}}
                \item {(x,y,z,1) \rl{برای نقطه ها}}
            \end{enumerate}
        }

        بعداً خواهیم دید که تنظیم $w=1$ برای نقاط به انتقال نقاط اجازه می دهد تا به درستی کار کنند
        و تنظیم $w=0$  برای بردارها از تغییر مختصات بردارها توسط انتقال ها جلوگیری می کند
        (ما نمی خواهیم مختصات یک بردار را انتقال دهیم که جهت و بزرگی آن را تغییر دهد - انتقال ها نباید خواص بردارها را تغییر دهند).

        \begin{point}{pnt:3.1}
            \Large
            علامت گذاری مختصات همگن با ایده های نشان داده شده در شکل \label{fig:4.Session.1.1.17} مطابقت دارد.
            یعنی تفاوت بین دو نقطه $\textbf{q-p}=(q_{x},q_{y},q_{z},1)\textbf{-}(p_{x},p_{y},p_{z},1)=(q_{x}-p_{x},q_{y}-p_{y},q_{z}-p_{z},0)$ منجر به بردار،
            و یک نقطه به اضافه یک بردار $\textbf{p+v}=(q_{x},q_{y},q_{z},1)\textbf{+}(v_{x},v_{y},v_{z},1)=(q_{x}+v_{x},q_{y}+v_{y},q_{z}+v_{z},1)$ منجر به نقطه میشود.
        \end{point}
    \end{spacing}
}

\subsection{\textbf{تعریف و نمایش ماتریس}}
{
    \Large
    \begin{spacing}{1.5}
        یک تبدیل خطی نمی تواند تمام تبدیل هایی را که ما می خواهیم انجام دهیم را توصیف کند.
        بنابراین، ما به یک کلاس بزرگتر از توابع به نام تبدیل های آفین تقویت میکنیم.
        تبدیل آفین یک تبدیل خطی به اضافه یک بردار انتقال $\textbf{b}$ است. به این معنا که:

        \begin{center}
            $\alpha(\textbf{u})=\tau(\textbf{u})+\textbf{b}$
        \end{center}

        یا در نماد ماتریسی:

        \begin{center}
            $\alpha(\textbf{u})=\textbf{uA}+\textbf{b}=[x, y, z]\begin{bmatrix}
                                                                    A_{11} & A_{12} & A_{13} \\
                                                                    A_{21} & A_{22} & A_{23} \\
                                                                    A_{31} & A_{32} & A_{33}
            \end{bmatrix}+[b_{x},b_{y},b_{z}]=[x\prime,y\prime,z\prime]$
        \end{center}

        که در آن $\textbf{A}$ نمایش ماتریسی یک تبدیل خطی است.

        اگر مختصات همگن را با $w=1$ افزایش دهیم، می‌توانیم این را فشرده‌تر بنویسیم:

        \begin{eqtn}{eqtn:3.6}
            \centering
            $[x, y, z, 1]\begin{bmatrix}
                             A_{11} & A_{12} & A_{13} & 0 \\
                             A_{21} & A_{22} & A_{23} & 0 \\
                             A_{31} & A_{32} & A_{33} & 0
                             b_{x} & b_{y} & b_{y} & 1
            \end{bmatrix}=[x\prime, y\prime, z\prime,1]$
        \end{eqtn}

        ماتریس $4\times 4$ در معادله \ref{eqtn:3.6} نمایش ماتریسی تبدیل آفین نامیده می شود.
        توجه کنید که جمع با $\textbf{b}$ اساساً یک انتقال است (یعنی تغییر در موقعیت).
        ما نمی خواهیم این را برای بردارها اعمال کنیم زیرا بردارها موقعیتی ندارند.
        با این حال، ما هنوز هم می خواهیم قسمت خطی تبدیل آفین را به بردارها اعمال کنیم.
        اگر $w=1$ را در جزء چهارم برای بردارها قرار دهیم، انتقال $\textbf{b}$ اعمال نمی شود (با انجام ضرب ماتریس بررسی کنید).

        \begin{point}{pnt:3.1}
            \Large
            از آنجا که ضرب داخلی بردار ردیف با ستون چهارم ماتریس تبدیل آفین $4\times 4$ فوق ، برابر است با:
            $[x, y, z, w]\cdot[0,0,0,1]=w$، این ماتریس مختصات $w$ بردار ورودی را تغییر نمی دهد.
        \end{point}
    \end{spacing}
}

\subsection{\textbf{انتقال}}
{
    \Large
    \begin{spacing}{1.5}
        تبدیل همانی یک تبدیل خطی است که فقط آرگومان خود را برمی‌گرداند.
        یعنی $I\textbf{u}=\textbf{u}$.
        می توان نشان داد که نمایش ماتریسی این تبدیل خطی، ماتریس همانی است.

        حال، تبدیل انتقال را به صورت تبدیل آفین تعریف می کنیم که تبدیل خطی آن تبدیل همانی است. به این معنا که،

        \begin{center}
            $\tau(\textbf{u})=\textbf{uI}+\textbf{b}=\textbf{u}+\textbf{b}$
        \end{center}

        \begin{figure}[H]
            \centering
            \setlength{\belowcaptionskip}{-10pt}
            \includegraphics[width=0.5\textwidth]{Images/4/3/4.Session.1.3.5}
            \caption {جابجایی موقعیت مورچه با مقدار بردار جابجایی $\textbf{b}$.}
            \label{fig:4.Session.1.3.5}
        \end{figure}

        همانطور که می بینید، این به سادگی نقطه $\textbf{u}$ را با $\textbf{b}$ انتقال میدهد (یا جابجا می کند).
        شکل \ref{fig:4.Session.1.3.5} نشان می دهد که چگونه می توان از آن برای جابجایی اشیا استفاده کرد - ما هر نقطه روی شی را با همان بردار $\textbf{b}$ انتقال میدهیم تا آن را جابجا کنیم.
        با معادله \ref{eqtn:3.6}، $\tau$ نمایش ماتریسی زیر را دارد:

        \begin{center}
            $\textbf{T}=\begin{bmatrix}
                            1 & 0 & 0 & 0 \\
                            0 & 1 & 0 & 0 \\
                            0 & 0 & 1 & 0
                            b_{x} & b_{y} & b_{y} & 1
            \end{bmatrix}$
        \end{center}

        به این ماتریس انتقال می گویند.

        معکوس ماتریس انتقال به صورت زیر بدست می آید:

        \begin{center}
            $\textbf{T}^{-1}=\begin{bmatrix}
                                 1 & 0 & 0 & 0 \\
                                 0 & 1 & 0 & 0 \\
                                 0 & 0 & 1 & 0
                                 -b_{x} & -b_{y} & -b_{y} & 1
            \end{bmatrix}$
        \end{center}

        \begin{example}{exp:3.4}
            \Large
            فرض کنید مربعی داریم که با یک نقطه حداقل $(-8, 2, 0)$ و یک نقطه حداکثر $(-2, 8, 0)$ تعریف شده است.
            اکنون فرض کنید که می‌خواهیم مربع را $12$ واحد در محور $x$، $-10$ واحد در محور $y$ انتقال دهیم
            و محور $z$ را بدون تغییر رها کنیم.
            ماتریس انتقال مربوطه به صورت زیر است:

            \begin{center}
                $\textbf{T}^{-1}=\begin{bmatrix}
                                     1 & 0 & 0 & 0 \\
                                     0 & 1 & 0 & 0 \\
                                     0 & 0 & 1 & 0
                                     12 & -10 & 0 & 1
                \end{bmatrix}$
            \end{center}

            اکنون برای انتقال (تبدیل) مربع، هر دو نقطه حداقل و حداکثر را در این ماتریس ضرب می کنیم:

            \begin{center}
                $[-8, 2, 0, 1]\begin{bmatrix}
                                  1 & 0 & 0 & 0 \\
                                  0 & 1 & 0 & 0 \\
                                  0 & 0 & 1 & 0
                                  12 & -10 & 0 & 1
                \end{bmatrix}=[4, -8, 0, 1]$ \\
                $[-2, 8, 0, 1]\begin{bmatrix}
                                  1 & 0 & 0 & 0 \\
                                  0 & 1 & 0 & 0 \\
                                  0 & 0 & 1 & 0
                                  12 & -10 & 0 & 1
                \end{bmatrix}=[10, -2, 0, 1]$
            \end{center}

            نتیجه در شکل \ref{fig:4.Session.1.3.6} نشان داده شده است.

            \begin{figure}[H]
                \centering
                \setlength{\belowcaptionskip}{-10pt}
                \includegraphics[width=0.8\textwidth]{Images/4/3/4.Session.1.3.6}
                \caption {انتقال $12$ واحد در محور $x$ و $-10$ واحد در محور $y$.
                توجه داشته باشید که هنگام نگاه کردن به محور $z$ منفی، هندسه اساساً 2 بعدی است زیرا $z=0$ است.}
                \label{fig:4.Session.1.3.6}
            \end{figure}
        \end{example}

        \begin{point}{pnt:3.1}
            \Large
            بگذارید $\textbf{T}$ یک ماتریس تبدیل باشد و به یاد بیاورید که با محاسبه حاصلضرب $\textbf{vT=v\prime}$ یک نقطه/بردار را تبدیل می کنیم.
            توجه کنید که اگر یک نقطه/بردار را با $\textbf{T}$ تبدیل کنیم و سپس دوباره آن را با $\textbf{T}^{-1}$ معکوس تبدیل کنیم،
            به بردار اصلی می‌رسیم: $\textbf{vTT^{-1}=vI=v}$.
            به عبارت دیگر، تبدیل معکوس ، تبدیل را خنثی می کند.
            به عنوان مثال، اگر یک نقطه را $5$ واحد در محور $x$ انتقال داده، و سپس برعکس $-5$ واحد در محور $x$ انتقال دهیم، به جایی می رسیم که شروع کرده ایم.
            به همین ترتیب، اگر نقطه ای را $30^\circ$ درجه حول محور $y$ دوران دهیم، و سپس برعکس $-30^\circ$ درجه حول محور $y$ دوران دهیم، در نهایت به نقطه اصلی خود می رسیم.
            به طور خلاصه، معکوس یک ماتریس تبدیل، تبدیل مخالف را انجام می دهد، به طوری که ترکیب دو تبدیل، هندسه را بدون تغییر می گذارد.
        \end{point}
    \end{spacing}
}

\subsection{\textbf{ماتریس های وابسته برای مقیاس بندی و دوران}}
{
    \Large
    \begin{spacing}{1.5}
        مشاهده کنید که اگر $\textbf{b}=0$ باشد، تبدیل آفین به تبدیل خطی کاهش می یابد.
        بنابراین ما می‌توانیم هر تبدیل خطی را به‌عنوان یک تبدیل آفین با $\textbf{b}=0$ بیان کنیم.
        این به نوبه خود به این معنی است که می‌توانیم هر تبدیل خطی را با یک ماتریس آفین $4\times 4$ نشان دهیم.
        به عنوان مثال، ماتریس های مقیاس بندی و دوران نوشته شده با استفاده از ماتریس $4\times 4$ به صورت زیر آورده شده است:

        \begin{center}
            $\textbf{S}=\begin{bmatrix}
                            s_x & 0   & 0   & 0 \\
                            0   & s_y & 0   & 0 \\
                            0   & 0   & s_z & 0 \\
                            0   & 0   & 0   & 1
            \end{bmatrix}$ \\
            \begin{bmatrix}
                c+(1-c)x^{2} & (1-c)xy+sz & (1-c)xz-sy & 0\\
                (1-c)xy-sz & c+(1-c)y^{2} & (1-c)yz+sx & 0\\
                (1-c)xz+sy & (1-c)yz-sx & c+(1-c)z^{2} & 0
                0 & 0 & 0 & 1
            \end{bmatrix}
        \end{center}

        به این ترتیب، می‌توانیم تمام تبدیل‌های خود را با استفاده از ماتریس $4\times 4$ و نقطه و بردار با استفاده از بردار ردیف همگن $4\times 1$ بیان کنیم.
    \end{spacing}
}

\subsection{\textbf{تفسیر هندسی یک ماتریس تبدیل وابسته}}
{
    \Large
    \begin{spacing}{1.5}
        در این بخش، شهودی از معنای هندسی اعداد درون ماتریس تبدیل آفین ایجاد می کنیم.
        ابتدا، اجازه دهید یک تبدیل جسم صلب را در نظر بگیریم، که اساساً یک تبدیل حفظ شکل است.
        یک مثال در دنیای واقعی از تبدیل جسم صلب ممکن است برداشتن یک کتاب از روی میز و قرار دادن آن در قفسه کتاب باشد.
        در طول این فرآیند شما کتاب را از روی میز خود به قفسه کتاب انتقال میدهید، اما به احتمال زیاد جهت کتاب را در فرآیند (دوران) تغییر می دهید.
        فرض کنید $\tau$ یک تبدیل دورانی باشد که نحوه دوران یک شی را توضیح می دهد و اجازه دهید $\textbf{b}$ یک بردار جابجایی تعریف کند که توضیح می دهد چگونه می خواهیم یک شی را انتقال دهیم. این تبدیل جسم صلب را می توان با تبدیل آفین توصیف کرد:

        \begin{center}
            $\alpha(x,y,z)=\tau(x,y,z)+\textbf{b}=x\tau(\textbf{i})+y\tau(\textbf{j})+z\tau(\textbf{k})+\textbf{b}$
        \end{center}

        در نمادگذاری ماتریسی، با استفاده از مختصات همگن ($w=1$ برای نقاط و $w=0$ برای بردارها به طوری که انتقال برای بردارها اعمال نشود)، به صورت زیر نوشته می شود:

        \begin{eqtn}{eqtn:3.7}
            \centering
            $[x,y,z,w]\begin{bmatrix}
                          \leftarrow & \tau(\textbf{i}) & \rightarrow \\
                          \leftarrow & \tau(\textbf{j}) & \rightarrow \\
                          \leftarrow & \tau(\textbf{k}) & \rightarrow \\
                          \leftarrow & \textbf{b}       & \rightarrow
            \end{bmatrix}=[x\prime,y\prime,z\prime,w\prime]$
        \end{eqtn}

        اکنون، برای اینکه ببینیم این معادله از نظر هندسی چه کاری انجام می دهد، تنها کاری که باید انجام دهیم این است که بردارهای ردیف را در ماتریس نمودار کنیم (شکل \ref{fig:4.Session.1.3.7} را ببینید).
        از آنجا که $\tau$ یک تبدیل دورانی است، طول ها و زوایا را حفظ می کند.
        به طور خاص، می بینیم که $\tau$ فقط بردارهای پایه استاندارد $i$، $j$، و $k$ را به یک جهت جدید $\tau(\textbf{i})$)، $\tau(\textbf{j})$) و $\tau(\textbf{k})$) دوران میدهد.
        بردار $\textbf{b}$ فقط یک بردار موقعیت است که نشان دهنده جابجایی از مبدا است.
        اکنون شکل \ref{fig:4.Session.1.3.7} نشان می دهد که چگونه نقطه تبدیل شده از نظر هندسی زمانی که $\alpha(x,y,z)=x\tau(\textbf{i})+y\tau(\textbf{j})+z\tau(\textbf{k})+\textbf{b}$ محاسبه می شود از نظر هندسی به دست می آید.
        همین ایده برای مقیاس بندی یا تغییر شکل‌ها همانطور که در شکل \ref{fig:4.Session.1.3.8} نشان داده شده است،
        تبدیل خطی $\tau$ را در نظر بگیرید که یک مربع را به متوازی الاضلاع تبدیل می کند.
        نقطه تاب دار به سادگی ترکیب خطی بردارهای پایه تاب شده است.

        \begin{figure}[H]
            \centering
            \setlength{\belowcaptionskip}{-10pt}
            \includegraphics[width=\textwidth]{Images/4/3/4.Session.1.3.7}
            \caption {هندسه سطرهای یک ماتریس تبدیل آفین. نقطه تبدیل شده، $\alpha(\textbf{p})$)، به صورت ترکیبی خطی از بردارهای مبنا تبدیل شده \tau(\textbf{i})، \tau(\textbf{j}) و \tau(\textbf{k}) و آفست $\textbf{b}$ داده می شود.}
            \label{fig:4.Session.1.3.7}
        \end{figure}

        \begin{figure}[H]
            \centering
            \setlength{\belowcaptionskip}{-10pt}
            \includegraphics[width=0.8\textwidth]{Images/4/3/4.Session.1.3.8}
            \caption {برای تبدیل خطی که مربع را به متوازی الاضلاع می پیچد، نقطه تبدیل شده $\alpha(\textbf{p})=(x,y)$ به صورت ترکیبی خطی از بردارهای مبنا تبدیل شده \tau(\textbf{i}) و \tau(\textbf{j}) داده می شود.}
            \label{fig:4.Session.1.3.8}
        \end{figure}
    \end{spacing}
}


\section{\textbf{ترکیب تبدیلات}}
\label{sec:3.3}
{
    \Large
    \begin{spacing}{1.5}
        فرض کنید $\textbf{S}$ یک ماتریس مقیاس بندی، $\textbf{R}$ یک ماتریس دوران و $\textbf{T}$ یک ماتریس انتقال است.
        فرض کنید مکعبی داریم که از هشت رأس $\textbf{v}_{i}$ به ازای $i=0,1,...,7$ تشکیل شده است و
        می‌خواهیم این سه تبدیل را به صورت متوالی در هر راس اعمال کنیم. روش واضح برای انجام این کار گام به گام است:

        \begin{center}
            $((\textbf{v}_{i}\textbf{S})\textbf{R})\textbf{T}=(\textbf{v}\prime_{i}\textbf{R})\textbf{T}=\textbf{v}\prime\prime_{i}\textbf{T}=\textbf{v}\prime\prime\prime_{i}\hspace(5 mm)for i=0,1,\dots,7$
        \end{center}

        با این حال، از آنجایی که ضرب ماتریس شرکت پذیر است، در عوض می‌توانیم آن را به صورت معادل بنویسیم:

        \begin{center}
            $\textbf{v}_{i}(\textbf{SRT})=\textbf{v}\prime\prime\prime_{i}\hspace(5 mm)for i=0,1,\dots,7$
        \end{center}

        ما می توانیم ماتریس $\textbf{C}=\textbf{SRT}$ را به عنوان ماتریسی در نظر بگیریم که هر سه تبدیل را در یک ماتریس تبدیل خالص محصور می کند.
        به عبارت دیگر، ضرب ماتریس-ماتریس به ما اجازه می دهد تا تبدیل ها را به هم متصل کنیم.

        این پیامدهای عملکردی دارد. برای مشاهده این موضوع، فرض کنید یک شی سه بعدی از $20000$ نقطه تشکیل شده است و ما می خواهیم این سه تبدیل هندسی متوالی را روی جسم اعمال کنیم.
        با استفاده از روش گام به گام، به $20000\times 3$ ضرب ماتریس برداری نیاز داریم.
        از سوی دیگر، استفاده از رویکرد ماتریس ترکیبی به $20000$ ضرب بردار-ماتریس و $2$ ضرب ماتریس-ماتریس نیاز دارد. واضح است که دو ضرب اضافی ماتریس-ماتریس هزینه ارزانی برای صرفه جویی زیاد در ضرب ماتریس بردار است.

        \begin{point}{pnt:3.1}
            \Large
            مجدداً اشاره می کنیم که ضرب ماتریس جابجایی پذیر نیست. این حتی به صورت هندسی نیز دیده می شود.
            به عنوان مثال، یک دوران و سپس یک انتقال، که می‌توانیم آن را با حاصلضرب ماتریس $\textbf{RT}$ توصیف کنیم، به همان تبدیلی که همان انتقال به دنبال آن دوران یکسان است، یعنی $\textbf{TR}$، منجر نمی‌شود. شکل \ref{fig:4.Session.1.3.9} این را نشان می دهد.
        \end{point}

        \begin{figure}[H]
            \centering
            \setlength{\belowcaptionskip}{-10pt}
            \includegraphics[width=0.8\textwidth]{Images/4/3/4.Session.1.3.9}
            \caption {(الف) ابتدا دوران و سپس انتقال. (ب) ابتدا انتقال و سپس دوران.}
            \label{fig:4.Session.1.3.9}
        \end{figure}
    \end{spacing}
}


\section{\textbf{تغییر تبدیلات مختصات}}
\label{sec:3.4}
{
    \Large
    \begin{spacing}{1.5}
        اسکالر $100$ درجه سانتیگراد نشان دهنده دمای آب در حال جوش نسبت به مقیاس سانتیگراد است.
        چگونه دمای یکسان آب جوش را نسبت به مقیاس فارنهایت توصیف کنیم؟ به عبارت دیگر، اسکالر نسبت به مقیاس فارنهایت که دمای آب در حال جوش را نشان می دهد چیست؟ برای انجام این تبدیل (یا تغییر فریم)، ​​باید بدانیم مقیاس های سانتیگراد و فارنهایت چگونه با هم ارتباط دارند.

        آنها به شرح روبرو مرتبط هستند:
        $T_{F}=\frac{\displaystyle 9}{\displaystyle 5}T_{C}+32^\circ$. بنابراین دمای آب در حال جوش نسبت به مقیاس فارنهایت با$T_{F}=\frac{\displaystyle 9}{\displaystyle 5}(100)^\circ+32^\circ=212^\circ F$ به دست می آید.

        این مثال نشان می‌دهد که ما می‌توانیم یک اسکالر $k$ که مقداری را نسبت به یک سیستم مختصات $A$ توصیف می‌کند،
        به یک اسکالر جدید $k\prime$ تبدیل کنیم که همان کمیت را نسبت به یک سیستم مختصات دیگر $B$ توصیف می‌کند،
        مشروط بر اینکه بدانیم سیستم مختصات $A$ و  $B$ چگونه به هم مرتبط هستند.
        در بخش‌های فرعی بعدی، ما به یک مشکل مشابه نگاه می‌کنیم، اما به جای اسکالرها، به چگونگی تبدیل مختصات یک نقطه/بردار نسبت به یک سیستم مختصات به مختصات نسبت به یک سیستم مختصات متفاوت علاقه‌مندیم (شکل \ref{fig:4.Session.1.3.10} را ببینید).
        تبدیل مختصات را از یک سیستم مختصات به مختصات سیستم مختصات دیگر را تبدیل مختصات می نامیم.
        شایان ذکر است که در تغییر تبدیل مختصات، هندسه را تغییر نمیدهیم. در عوض، ما چارچوب مرجع را تغییر می دهیم، که بنابراین نمایش مختصات هندسه را تغییر می دهد.
        این برخلاف طرز فکر ما در مورد دوران ها، انتقال ها و مقیاس بندی است، که در آن به حرکت فیزیکی یا تغییر شکل هندسه فکر می کنیم.
        در گرافیک کامپیوتری سه بعدی، ما از چند سیستم مختصات استفاده می کنیم، بنابراین باید بدانیم که چگونه از یک سیستم تبدیل کنیم.
        به دیگری. از آنجایی که مکان ویژگی نقاط است، اما نه بردارها، تغییر تبدیل مختصات برای نقاط و بردارها متفاوت است.

        \begin{figure}[H]
            \centering
            \setlength{\belowcaptionskip}{-10pt}
            \includegraphics[width=0.8\textwidth]{Images/4/3/4.Session.1.3.10}
            \caption {همان بردار $\textbf{v}$ زمانی که نسبت به سیستم متخصات های مختلف توضیح داده می شود مختصات متفاوتی دارد. مختصات $(x, y)$ نسبت به سیستم متخصات $A$ و مختصات $(x\prime, y\prime)$ نسبت به سیستم متخصات $B$ دارد.}
            \label{fig:4.Session.1.3.10}
        \end{figure}
    \end{spacing}
}

\subsection{\textbf{بردار ها}}
{
    \Large
    \begin{spacing}{1.5}
        شکل \ref{fig:4.Session.1.3.11} را در نظر بگیرید که در آن دو سیستم مختصات $A$ و $B$ و یک بردار $\textbf{p}$ داریم.
        فرض کنید مختصات $\textbf{p}$، $\textbf{p}_{B}=(x, y)$ نسبت به سیستم مختصات $A$ به ما داده شده است
        و می خواهیم مختصات $\textbf{p}$، $\textbf{p}_{B}=(x\prime, y\prime)$ را نسبت به سیستم مختصات $B$ پیدا کنیم.
        به عبارت دیگر، با توجه به مختصات با شناسایی یک بردار نسبت به یک سیستم مختصات، چگونه مختصاتی را پیدا کنیم که همان بردار را نسبت به یک سیستم مختصات دیگر مشخص می کند؟

        \begin{figure}[H]
            \centering
            \setlength{\belowcaptionskip}{-10pt}
            \includegraphics[width=0.8\textwidth]{Images/4/3/4.Session.1.3.11}
            \caption {هندسه یافتن مختصات $\textbf{p}$ نسبت به سیستم مختصات $B$.}
            \label{fig:4.Session.1.3.11}
        \end{figure}

        از شکل \ref{fig:4.Session.1.3.11}، مشخص است که

        \begin{center}
            $\textbf{p}=x\textbf{u}+y\textbf{v}$
        \end{center}

        که در آن $\textbf{u}$ و $\textbf{v}$ بردارهای واحدی هستند که به ترتیب در امتداد محورهای $x$ و $y$ سیستم مختصات $A$ هدف قرار می گیرند.
        با بیان هر بردار در معادله بالا در مختصات سیستم مختصات $B$ به دست می آوریم:

        \begin{center}
            $\textbf{p}_{B}=x\textbf{u}_{B}+y\textbf{v}_{B}$
        \end{center}

        بنابراین، اگر $\textbf{p}_{A}=(x, y)$ به ما داده شود و مختصات بردارهای $\textbf{u}$ و $\textbf{v}$ را نسبت به سیستم مختصات $B$ بدانیم،
        یعنی اگر $\textbf{u}_{B}=(u_{x}, u_{y})$ و $\textbf{v}_{B}=(v_{x}, v_{y})$ را بدانیم، سپس ما همیشه می توانیم $\textbf{p}_{B}=(x\prime, y\prime)$ را پیدا کنیم.

        تعمیم به سه بعدی، اگر $\textbf{p}_{A}=(x, y, z)$، پس

        \begin{center}
            $\textbf{p}_{B}=x\textbf{u}_{B}+y\textbf{v}_{B}+z\textbf{w}_{B}$
        \end{center}

        که در آن $\textbf{u}$، $\textbf{v}$ و $\textbf{w}$ بردارهای واحدی هستند که به ترتیب در امتداد محورهای $x$، $y$ و $z$ سیستم متخصات $A$ هدف قرار می‌گیرند.
    \end{spacing}
}

\subsection{\textbf{نقاط}}
{
    \Large
    \begin{spacing}{1.5}
        تغییر تبدیل مختصات برای نقاط کمی متفاوت از آن برای بردار است.
        این به این دلیل است که مکان برای نقاط مهم است، بنابراین ما نمی‌توانیم نقاط را همانطور که بردارهای شکل 3.11 را نتقال دادیم، انتقال دهیم.

        \begin{figure}[H]
            \centering
            \setlength{\belowcaptionskip}{-10pt}
            \includegraphics[width=0.8\textwidth]{Images/4/3/4.Session.1.3.12}
            \caption {هندسه یافتن مختصات $\textbf{p}$ نسبت به سیستم مختصات $B$.}
            \label{fig:4.Session.1.3.12}
        \end{figure}

        شکل \ref{fig:4.Session.1.3.12} وضعیت را نشان می دهد و می بینیم که نقطه $\textbf{p}$ را می توان با معادله بیان کرد:

        \begin{center}
            $\textbf{p}=x\textbf{u}+y\textbf{v}+\textbf{Q}$
        \end{center}

        که در آن $\textbf{u}$ و $\textbf{v}$ بردارهای واحدی هستند که به ترتیب در امتداد محورهای $x$ و $y$ سیستم متخصات $A$ هدف قرار می گیرند و $\textbf{Q}$ مبدا سیستم مختصات $A$ است.
با بیان هر بردار/نقطه در معادله بالا در مختصات سیستم مختصات $B$ به دست می آوریم:

        \begin{center}
            $\textbf{p}_{B}=x\textbf{u}_{B}+y\textbf{v}_{B}+\textbf{Q}_{B}$
        \end{center}

        بنابراین، اگر $\textbf{p}_{A}=(x, y)$ به ما داده شود و مختصات بردارهای $\textbf{u}$ و $\textbf{v}$ و مبدأ $\textbf{Q}$ را نسبت به سیستم مختصات $B$ بدانیم، یعنی اگر $\textbf{u}_{B}=(u_{x}, u_{y})$ ، $\textbf{v}_{B}=(v_{x}, v_{y})$ و $\textbf{Q}_{B}=(Q_{x}, Q_{y})$ را بدانیم، سپس ما همیشه می توانیم $\textbf{p}_{B}=(x\prime, y\prime)$ را پیدا کنیم.

تعمیم به سه بعدی، اگر $\textbf{p}_{A}=(x, y, z)$، پس

        \begin{center}
            $\textbf{p}_{B}=x\textbf{u}_{B}+y\textbf{v}_{B}+z\textbf{w}_{B}+\textbf{Q}_{B}$
        \end{center}

        که در آن $\textbf{u}$، $\textbf{v}$ و $\textbf{w}$ بردارهای واحدی هستند که به ترتیب در امتداد محورهای $x$، $y$ و $z$ سیستم متخصات $A$ هدف قرار می گیرند و $\textbf{Q}$ مبدا سیستم متخصات $A$ است.
    \end{spacing}
}

\subsection{\textbf{نمایش ماتریسی}}
{
    \Large
    \begin{spacing}{1.5}
        \begin{figure}[H]
            \centering
            \setlength{\belowcaptionskip}{-10pt}
            \includegraphics[width=0.8\textwidth]{Images/4/3/4.Session.1.3.12}
            \caption {خروجی برنامه ی بالا.}
            \label{fig:4.Session.1.3.12}
        \end{figure}

        \begin{eqtn}{eqtn:3.8}
            \centering
        \end{eqtn}

        \begin{eqtn}{eqtn:3.9}
            \begin{equation*}
                \centering
                \begin{split}
                    \textbf{AB}&=
                    &=
                \end{split}
            \end{equation*}
            \centering
        \end{eqtn}

    \end{spacing}
}

\subsection{\textbf{شرکت پذیری و تغییر ماتریس مختصات}}
{
    \Large
    \begin{spacing}{1.5}
        \begin{center}
        \end{center}

        \begin{point}{pnt:3.1}
            \Large
        \end{point}

    \end{spacing}
}

\subsection{\textbf{معکوس ها و تغییر ماتریس های مختصات}}
{
    \Large
    \begin{spacing}{1.5}
        \begin{flushleft}
        \end{flushleft}

        \begin{figure}[H]
            \centering
            \setlength{\belowcaptionskip}{-10pt}
            \includegraphics[width=0.5\textwidth]{Images/4/3/4.Session.1.3.13}
            \caption {خروجی برنامه ی بالا.}
            \label{fig:4.Session.1.3.13}
        \end{figure}

        \begin{figure}[H]
            \centering
            \setlength{\belowcaptionskip}{-10pt}
            \includegraphics[width=0.8\textwidth]{Images/4/3/4.Session.1.3.14}
            \caption {خروجی برنامه ی بالا.}
            \label{fig:4.Session.1.3.14}
        \end{figure}

    \end{spacing}
}

\subsection{\textbf{معکوس ها در مقابل تغییر ماتریس های مختصات}}
{
    \Large
    \begin{spacing}{1.5}
        \begin{flushleft}
        \end{flushleft}

        \begin{figure}[H]
            \centering
            \setlength{\belowcaptionskip}{-10pt}
            \includegraphics[width=0.5\textwidth]{Images/4/3/4.Session.1.3.15}
            \caption {خروجی برنامه ی بالا.}
            \label{fig:4.Session.1.3.13}
        \end{figure}

        \begin{point}{pnt:3.1}
            \Large
        \end{point}

    \end{spacing}
}


\section{\textbf{توابع تبدیل ریاضی \lr{DirectX}}}
\label{sec:3.5}
{
    \Large
    \begin{spacing}{1.5}

        \textbf{\vspace{6pt}}
        \lr{\lstinputlisting[language=C++, firstline=1, lastline=30]{Codes/4.1.3.program.c}}
        \textbf{\vspace{6pt}}

        \texttt{XMMATRIX}

    \end{spacing}
}
%-----------------------------------------------------------------------------------------------------------%
\newpage


\section{\textbf{خلاصه}}
\label{sec:3.6}
{
    \Large
    \begin{spacing}{1.5}
        \begin{enumerate}[label=\textbf{\arabic*}.]
            \item {}

            \item {}

            \item {}

            \item {}

            \item {}

            \item {}

            \item {}

            \item {}
        \end{enumerate}
    \end{spacing}
}
%-----------------------------------------------------------------------------------------------------------%
\newpage


\section{\textbf{تمارین}}
\label{sec:3.7}
{
    \Large
    \begin{spacing}{1.5}
        \begin{enumerate}[label=\textbf{\arabic*}.]
            \item {}

            \item {}

            \item {}

            \item {}

            \item {}

            \item {}

            \item {}

            \item {}

            \item {}

            \item {}

            \item {}

            \item {}

            \item {}

            \item {}

            \item {
                \begin{center}
                \end{center}
            }

            \item {}

            \item {}

            \item {
                \begin{center}
                \end{center}
            }

            \item {
                \lr{
                    \begin{flushleft}
                    (a)
                        $\textbf{v}=2(1,2,3)-4(-5,0,-1)+3(2,-2,3)$ \\
                        (b) $\textbf{v}=2(2,-4)+2(1,4)-1(-2,-3)+5(1,1)$
                    \end{flushleft}
                } \textbf{\vspace{-6pt}}
            }

            \item {}

            \item {
                \begin{center}
                \end{center}
                \begin{figure}[H]
                    \centering
                    \setlength{\belowcaptionskip}{-10pt}
                    \includegraphics[width=0.8\textwidth]{Images/4/3/4.Session.1.3.16}
                    \caption {خروجی برنامه ی بالا.}
                    \label{fig:4.Session.1.3.16}
                \end{figure}
            }

            \item {}

            \item {
                \begin{center}
                \end{center}
                \begin{figure}[H]
                    \centering
                    \setlength{\belowcaptionskip}{-10pt}
                    \includegraphics[width=0.8\textwidth]{Images/4/3/4.Session.1.3.16}
                    \caption {خروجی برنامه ی بالا.}
                    \label{fig:4.Session.1.3.16}
                \end{figure}
            }

            \item {}

            \item {}

            \item {
                \lr{
                    \begin{flushleft}
                    (a)
                        $\textbf{v}=2(1,2,3)-4(-5,0,-1)+3(2,-2,3)$ \\
                        (b) $\textbf{v}=2(2,-4)+2(1,4)-1(-2,-3)+5(1,1)$
                    \end{flushleft}
                    \begin{center}
                    \end{center}
                } \textbf{\vspace{-6pt}}
            }

            \item {}

            \item {
                \begin{hint}{hnt:2.1}
                    \Large

                \end{hint} \\

                \begin{figure}[H]
                    \centering
                    \setlength{\belowcaptionskip}{-10pt}
                    \includegraphics[width=0.8\textwidth]{Images/4/3/4.Session.1.3.18}
                    \caption {خروجی برنامه ی بالا.}
                    \label{fig:4.Session.1.3.18}
                \end{figure}
            }
        \end{enumerate}
    \end{spacing}
}